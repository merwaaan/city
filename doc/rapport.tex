\documentclass[12pt]{article}

\usepackage[utf8]{inputenc}
\usepackage[francais]{babel}

\usepackage{amsmath}

\usepackage{caption}
\usepackage{subcaption}

\usepackage{tikz}

\usetikzlibrary{calc}

\usepackage[natbib=true,sorting=none,style=numeric,url=false,doi=false,isbn=false]{biblatex}
%\addbibresource{references}
\bibliography{references}

\begin{document}

\begin{titlepage}

  Stage de recherche

  {\Large \textsc{Coévolution du réseau viaire et du bâti}}

  \begin{flushright}
    \textit{Auteur :}\\
    Merwan {\scshape Achibet}\\[0.5cm]
    \textit{Encadrants :}\\
    Stefan {\scshape Balev}\\
    Antoine {\scshape Dutot}\\
    Damien {\scshape Olivier}
  \end{flushright}

  \vfill

  \begin{center}
    ILLUSTRATION
  \end{center}

  \vfill

  \begin{center}
    \includegraphics[width=.25\linewidth]{images/logo-univ-le-havre.png}
    \qquad\qquad\qquad
    \includegraphics[width=.25\linewidth]{images/logo-litis.png}
  \end{center}

  \begin{center}
    {\small Mars - Juin 2012}
  \end{center}

\end{titlepage}

\begin{center}
  {\scshape\textbf Keywords}
\end{center}

{Urban system, city morphogenesis, Voronoi diagram, cellular automaton.}

\begin{center}
  {\scshape\textbf Extended abstract}
\end{center}

Gathering issues of human, economic, geographic and political nature,
the city truly is a complex system. The increasing growth in
population and technology produces urban systems like the world has
never seen before, of increasing size, increasing heterogeneity and
increasing complexity. Studying the relationships between its internal
elements is primordial to the understanding of its mechanics and to
better predict their developement. This work focuses on the
relationship between the pattern of human installations within the
city -- characterized at the atomic level by a basic subdivision, the
land lot -- and its road network.

Systematic definitions of the city are many. Through the
anthropologist's eye it can be seen as a concentration of persons, the
economist will prefer to view it as a support for the exchange of
financial and physical assets and the urbanist as a functional entity
composed of flows and services. We choose to consider that the
evolution of a city is driven by its population, translated as a
measure of its density. This goes in hand with our focus as land lots
are inhabited by the same people using the surrounding roads to go
from one place to another and urbanistic decisions are motivated by a
need to optimize the city, streamlining transport means and avoiding
high contrasts in the urban fabric.

Study of the domain of urban simulations shows that available
scientific works can be classified in two different categories. For
visual rendering purposes, special effects in movies or video games,
methods have been proposed that generate a visually satisfying system
without ??? realism. These are often based on empirical observations
and the main idea is to emulate the street patterns found in any urban
system. Scientific??? modeling, on the other hand, prefers to focus on
the inner qualities of a city, often by studying a subset of those, to
analyze its current state and extrapolate its future. We orient our
research with the latter in mind but the former represents a
non-negligible source of inspiration.

A reader browsing through the field litterature will often encounter
cellular automata. This structure, which applicability and potential
complexity has been proven over times, is fit for describing any kind
of space-related problem. Nevertheless, its rigorous formalism may
sometimes restrain realism and impact the validity of the model. For
example, a cellular automaton topology is, by definition, perfectly
regular and representing a city with a set of identical and aligned
cells seems like a coarse simplification. In the same way, the fact
that each cell has an identical neighborhood structure, the temporal
synchronism and the state discretization may be questionned. We take a
drastic step towards realism and embrace the spatial aspect of the
city by replacing the classic cellular automaton with a Voronoï
diagram following the same basic concepts. Each of its cell represents
a land lot and neighborhood relationships that are ruled by adjacency
determine its future state ; the regularity constraint is thus
relaxed. Voronoi edges delineate the Voronoi space of land lots and,
as such, are perfect support for roads. +++

The different elements represented in our model (\textit{i. e. lots
  and roads}) are declined in two flavors???. \textit{Potential}
elements have a fuzzy??? status ; their spatial characteristics are
known but until they are definitely built, they have no effect on the
overall city and only represent an idea, a possible
outcome. \textit{Built} elements were potential elements that have
been constructed. They form the physical city. Concisely, the gist of
the model is to add potential elements to the city and, only later,
chose which ones will be built and which ones will be forgotten. The
road network expand to accomodate ++++. This works has been divided
into four separate mechanisms for clarity purposes.

The cellular part of the model lets the inner variables of the city
vary based on the Voronoi tesselation. Here, only population density
is considered and a set of simple rules reproduces a gradient from
centers to exterieur?. The vertical growth assured by this mechanism
+++

The placement of new potential lots is handled by a general ??? based
on vector fields. Each data we want to consider is represented as a
vector field that will guide the seed for a new lot. For example, a
field escaping from high density areas is used to ensure urban
sprawl. Another points to the closest road such that new lots
extending the city remain ??? to its main axes. Then all vector fields
are summed up with varying coefficient. This general approach can
model any kind of guidance or constraint ; it is possible to use a
field repelling the seed from any obstacle (unbuildable areas, water).

Which lots should I build?

The road network expansion is fully based on the placement of lots,
potential or built, as their supports are the Voronoi cells. NETWORK
SIMPLEX

RESULTATS/MESURES

MINI CONC

\newpage

\tableofcontents

\newpage

REMERCIEMENTS

\newpage

\section{Introduction}

Avec ses enjeux économiques, humains, ???, ???, la ville est
véritablement un système complexe. La croissance des systèmes urbains
atteint aujourd'hui un stade imprécédé???. La taille et la complexité
des villes augmentent STATS. Il est nécéssaire d'étudier ces
changements pour comprendre les mécanismes sous-jacents à ce type de
système et pour prévoir leur évolution future. On se concentre ici sur
l'évolution jointe de deux ensembles majeurs du tissu urbain : le bâti
et le viaire.

DETAILS

Ces deux aspects de la ville semblent aller de paire puisque le bâti
abrite les population tandis que le viaire leur permet de se
déplacer. Il est ainsi naturel d'utiliser la densité de population
comme force de guidage à l'évolution de la ville.

DETAILS

La première partie présente un état de l'art de la modélisation de
systèmes urbains et se concentre particulièrement sur les méthodes à
base d'automates cellulaires tout en adoptant un point de vue
historique pour justifier l'adoption de cette structure. Le modèle
conçu dans le cadre de ce stage de recherche est ensuite présenté en
seconde partie. La troisième section aborde des problématiques
pratiques rencontrées lors de l'implémentation dudit modèle. Enfin,
des mesures diverses sont employées pour tester et valider ce travail
dans la quatrième partie.

\section{\'Etat de l'art}

\subsection{Automates cellulaires et simulation urbaine}

La modélisation de systèmes complexes est longtemps uniquement passée
par l'usage de méthodes mathématiques; typiquement, des systèmes
d'équations différentielles. Ces techniques permettent de décrire des
lois d'évolution et d'observer, ainsi que de prédire par
extrapolation, le comportement de phénomènes du réel. Dans le cas de
modèles prenant en compte un vaste jeu de paramètres, cette approche
peut néanmoins se révéler délicate à employer. Plus intrinsèquement,
même si une telle modélisation est basée sur des observations ancrées
dans la réalité, il s'agit d'une représentation conceptuelle d'un
problème et aucune mimique des mécaniques sous-jacentes ne s'opère.

Historiquement, les prémices de l'informatique moderne et d'un tout
autre paradigme de modélisation sont à attribuer aux esprits du milieu
du vingtième siècle. Alan Turing introduit en 1936 la machine éponyme
qui, bien que purement théorique, possède un module de contrôle ainsi
qu'une mémoire et peut donc exécuter une infinité d'algorithmes. Cette
démarche se démarque de l'approche mathématique et semble plus
humaine; on ne résout pas un problème en utilisant des fonctions
associant une quantité à un résultat mais on agit véritablement sur
ses données. L'idée de base de Turing était d'ailleurs d'assimiler le
fonctionnement de sa machine au travail d'une personne remplissant les
cases d'un tableau infini.

TRANSITION. Entraîné par cette mouvance procédurale et en réaction aux
réseaux de neurones de McCulloch et Pitts, John von Neumann et
Stanislaw Ulam joignent leurs travaux pour concevoir l'automate
cellulaire en 19??  CITATION WOLFRAM : un système comprenant un
ensemble d'automates à états spatialement localisés (typiquement sous
forme de grille) et interconnectés en fonction de leur proximité. Les
entrées de chaque automate correspondent alors aux états des automates
voisins et de cette organisation se dégagent de fortes relations
d'interdépendance. Le jeu de la vie de Conway en est un exemple
classique. La simplicité de ses règles, mise en contraste avec la
variété des configurations engendrées, témoigne de la richesse des
automates cellulaires \cite{Gardner1970}.

Les automates cellulaires ont depuis été extensivement étudiés et sont
appliqués à l'étude de nombreux phénomènes biologiques, physiques et
sociaux \cite{Ganguly2003}. La motivation d'Ulam lors de leur
conception était d'ailleurs de modéliser la croissance de cristaux. On
peut aussi citer en exemple la simulation de la dynamique de fluides
\cite{Frisch1986} et de la croissance de tumeurs
\cite{Kansal2000}. Leur caractère spatial laisse supposer qu'ils sont
particulièrement adaptés aux applications géographiques, et dans le
cadre de notre problématique, urbaines. Ils ne fûrent paradoxalement
pas immédiatement exploités à cet effet et c'est seulement suite à un
article de Waldo Tobler, en 1975, que le rapprochement entre les
automates cellulaires et le domaine de la géographie apparaît
clairement \cite{Tobler1975}. Sont ensuite publiés des travaux majeurs
appliquant l'automate cellulaire à eds problématiques géographiques
multi-échelle telles que l'évolution d'épidémies \cite{Fu2003} et la
ségrégation de population \cite{Schelling1969}.

\begin{figure}
  \centering
  \includegraphics[width=.6\linewidth]{images/schelling.png}
  \caption{Configuration produite par le modèle de Schelling. Chacune
    des deux couleurs représente une population différente.}
  \label{fig:schelling}
\end{figure}

Une idée très exploitée dans ce domaine est d'associer un potentiel de
transition à chaque cellule et ce, vers tous les états qu'elles
peuvent prendre. Dans les modèles déterministes, la transition vers
l'état à plus haut potentiel est appliquée tandis que dans les modèles
stochastiques, un tirage aléatoire biaisé est préféré. Le potentiel
d'une cellule à passer à un nouvel état est déterminé en fonction de
paramètres propres au modèle. Peuvent être pris en compte l'élévation
du terrain, la densité de population, la proximité d'axes routiers, la
proximité de centres urbains, l'âge des parcelle, leur valeur; en
fait, toute combinaison d'attributs relatifs à un réseau urbain. Par
exemple, dans une simulation représentant les différents types
d'usage, le passage d'une cellule à l'état \textit{résidentiel}
pourrait dépendre de la proximité des commerces et des routes et de
l'éloignement des zones industrielles. Bien sûr, un nombre élevé de
paramètres à prendre en compte requiert un couplage fin et l'impact de
chaque variable peut être pondéré. Puisque les variations
individuelles de paramètres n'émergent pas de manière transparente à
la surface de la simulation, les modèles urbains basés sur des
automates cellulaires doivent être finement calibrés et leur réalisme
est un défi en soi. Pour contourner ce problème, Yeh et Li prônent
l'usage d'un réseau de neurones pour pondérer chaque paramètre à
partir de l'analyse de données cartographiques historiques
\cite{Yeh2002}.

Il est important de noter que la simplicité du formalisme enveloppant
un automate cellulaire strict s'oppose à la qualité de la simulation,
notamment dans le cadre de modèles spécifiques
\cite{Torrens2001}. Dans ce cas, une prise de liberté quant aux
formalisme originel est autorisée, voire nécessaire, pour obtenir des
résultats satisfaisants \cite{White1998}.

La première limite que le formalisme cellulaire de base impose est la
discrétisation des états que chaque cellule peut adopter. Même si
cette caractéristique fait partie intégrante des particularités qui
confèrent aux automates cellulaires leur simplicité d'usage et
d'analyse, la description de quantités pouvant arborer un éventail
infini de valeurs est alors impossible. Plus concrètement, il est aisé
de catégoriser les cellules d'un espace selon le fait, par exemple,
qu'elles contiennent des installations humaines ou non (état booléen)
\cite{Benguigui2004,Cornu2008} ou de façon plus sophistiquée, en fonction
de leur type d'usage (\textit{résidentiel}, \textit{commercial} et
\textit{industriel} \cite{Lechner} et plus
\cite{Dubos-Paillard2003}). Représenter des quantités réelles et des
variations continues est moins aisé. Pour symboliser plus finement la
densité au c\oe ur d'un ensemble urbain, Semboloni utilise par exemple
un automate cellulaire de dimension trois dans lequel plus une pile de
cellules actives est haute et plus la zone représentée est peuplée
\cite{Semboloni2000}. Plus généralement, on peut s'autoriser à
représenter l'état d'une cellule par un vecteur contenant des valeurs
réelles; des règles de transitions adaptées et mesurées sont alors à
mettre en place.

L'homogénéité d'un automate cellulaire fait partie intégrante de sa
définition originelle : en mettant de côté l'état qu'elles adoptent,
toutes les cellules sont identiques en forme et en structure de
voisinage. Dans le cadre de notre problématique, cette approche est
limitante car, dans une ville, les parcelles ne sont que rarement
identiques et alignées. Similairement, la notion de voisinage est
clairement à redéfinir. Pour des problèmes classiques, les voisinages
de von Neumann et de Moore sont régulièrement utilisés mais la
relation par contiguité qu'ils décrivent ne convient pas à la
représentation des liens de dépendance à plus grande échelle se
développant dans un système urbain. Le positionnement d'un bâtiment
résidentiel dans une ville se base évidemment sur le voisinage direct
des zones envisagées (on veut ajouter une maison dans un quartier
résidentiel) mais il faut aussi prendre en compte les alentours plus
distants (la centrale thermique se trouvant à 500 mètres du site peut
poser problème). Une solution possible est d'étendre les voisinages de
von Neumann et de Moore tout en conservant leur forme caractéristique
CITATION. O'Sullivan a choisi de relaxer cette contrainte de
partionnement spatial régulier pour faire un pas dans la direction du
réalisme \cite{O'Sullivan2000,O'Sullivan2001} : conventionnellement,
une cellule d'automate correspond à un sous-espace urbain ou bien une
parcelle cadastrale mais dans chacun de ces cas le modèle se base
évidemment sur une simplification grossière de l'espace étudié. Il
décide donc de donner à chaque cellule les mêmes qualités topologiques
que les parcelles qu'elles représentent, même formes, même dimensions,
même coordonnées (mais peut-on encore parler de cellules ?). Une
variété de relations de voisinage sont alors envisageables, par
voisinage au sens propre, par distance dans un rayon d'influence, par
visibilité. L'éloignement du formalisme cellulaire est drastique car
la structure perd de son homogénéité (chaque cellule est différente),
la couverture de l'espace n'est plus complète (des vides entre les
cellules apparaissent) et le voisinage diffère lui aussi. Mais
TOPOLOGIE REELLE.

\begin{figure}
  \centering
  \includegraphics[width=.7\linewidth]{images/gca.png}
  \caption{Hoxton, un quartier de Londres, modélisé par l'automate
    cellulaire graphe de David O'Sullivan \cite{O'Sullivan2000}.}
  \label{fig:sullivan}
\end{figure}

\begin{figure}
  \begin{subfigure}[b]{.5\linewidth}
    \centering
    \input{images/CA-simple.tikz}
    \caption{Automate cellulaire classique.}
  \end{subfigure}
  \begin{subfigure}[b]{.5\linewidth}
    \centering
    \input{images/CA-graph.tikz}
    \caption{Automate cellulaire graphe.}
  \end{subfigure}
\end{figure}

Une prise de liberté quant à l'aspect temporel est aussi
envisageable. Un automate cellulaire strict est synchrone, i.e. les
changements d'état de toutes les cellules s'effectuent
simultanément. Si le choix était fait de mettre à jour chaque état de
façon asynchrone, le comportement de l'automate en serait lourdement
modifié. Par exemple, les qualités auto-réplicatives de certaines
entités du jeu de la vie ne seraient pas garanties. Il est pourtant
légitime de se questionner sur la validité d'un tel choix dans une
simulation urbaine, premièrement parce qu'une ville est un système
complexe et désordonné, deuxièmement parce les processus qui s'y
déroulent sont réglés sur différentes échelles temporelles.

Bien que les automates cellulaires soient couramment utilisés pour
simuler le traffic routier (dans leur version 1D CITATION ou 2D
\cite{Queloz1996}), ils s'accordent peu avec la construction même d'un
réseau viaire. Dans les simulations cellulaires urbaines, le
positionnement des routes a un impact sur le développement des
cellules puisque le viaire attire le bâti mais le réseau est souvent
fourni en entrée et reste fixe. Nous sommes amener à nous interroger
sur la capacité des automates cellulaires à modéliser le développement
routier. Les relations de proximité les caractérisant sont-elles
adaptées à la construction de structures dont l'échelle est celle de
la ville et non plus celle de la parcelle ? Est-il bénéfique de
représenter une route par un ensemble de cellules ou plutôt par une
entité unique ? PRECISER

\subsection{Approches alternatives}

Les automates cellulaires ne sont pas l'unique moyen de modéliser la
croissance urbaine. Plusieurs simulations existantes sont des systèmes
multi-agent \cite{Lechner2003,Lechner2004}. Dans ces cas, un agent est
assimilé à un promoteur immobilier et peut acheter des terres, les
vendre, les développer ou changer leur type. Les actions qu'il
entreprend sont évaluées en fonction de l'impact sur la ville
(changement de la valeur immobilière, avis de la population) et des
réglementations locales afin d'éviter toute configuration illégale.
Pour la construction du réseau routier, une solution est de mettre en
place, en plus des agents promoteurs, deux types d'agents
traceurs. Les \textit{extenders} parcourent toute la surface du
terrain à la recherche de bâtiments isolés puis tracent une route
jusqu'au réseau urbain. Les \textit{connectors} se déplacent
uniquement sur le réseau viaire et y raccordent les bâtiments non
connectés se trouvant dans leur rayon de détection
\cite{Lechner2003}. PAS LE BON ORDRE

D'autres solutions s'éloignant des systèmes complexes et penchant du
côté de la génération procédurale de contenu existent. Souvent, le
domaine d'application de telles méthodes est l'infographie, le cinéma
et le jeu vidéo et l'objectif est alors de construire de manière
automatique une ville visuellement réaliste sans se soucier de son
caractère fonctionnel. Usuellement, l'organisation parcellaire dépend
entièrement du réseau routier car la première étape est souvent de
générer un réseau viaire complet puis de placer le bâti en subdivisant
récursivement les niches vides formées par les voies. Dans Citygen
\cite{Kelly2006b}, un point $p$ de l'espace est aléatoirement choisi
puis on calcule un ensemble de plusieurs routes raccordant $p$ au
réseau routier existant en faisant varier leur déviation angulaire et
un paramètre de bruit; la route finale est celle pour laquelle la
variation d'altitude est la plus faible. ECHANTILLONAGE CityEngine
\cite{Parish2001} utilise un L-System dont les règles permettent de
reproduire les différents motifs quadrillés, radiaux et organiques que
l'on retrouve dans une ville. La nature récursive des L-Systems permet
à ces motifs de se combiner et d'apparaître à différents niveaux de
profondeur; les résultats sont saisissants (voir figure
\ref{fig:cityengine}). Dans une autre simulation, le tracé des routes
suit les \textit{hyperstreamlines} \cite{Chen2008} formées par un
champ de vecteurs. Ce champ est calculé par combinaison de plusieurs
autres champs de vecteurs, chacun représentant des contraintes
directionnelles particulières telles que les zones interdites (eau,
espaces verts), l'altitude et la densité de population. Ces techniques
sont intrinsèquement géométriques, et comme précisé plus haut, le
résultat est purement visuel, mais elles représentent une source
d'inspiration à ne pas négliger.

\begin{figure}
  \centering
  \includegraphics[width=.8\linewidth]{images/cityengine.png}
  \caption{CityEngine \cite{Parish2001}.}
  \label{fig:cityengine}
\end{figure}

Souvent, SPECIALISATION. L'un des seuls modèles gérant à la fois
l'évolution du réseau viaire et du bâti est présenté par Weber
\cite{Weber2009} et n'emploie pas d'automate cellulaire. Le principe
est le suivant : à chaque agrandissement du réseau urbain, on crée
plusieurs routes virtuelles en suivant des règles géométriques
précises (allongement des voies existantes, limitation du degré des
carrefours à 4, l'angle entre chaque rue tend vers 90 degrés). Parmi
les $n$ routes générées, une seule sera construite. Pour la choisir,
le traffic sur ces nouvelles routes est simulé par des agents piétons
et véhicules et l'on identifie celle qui sera la plus bénéfique au
réseau.

\section{Le modèle}

\subsection{Structure}

Les automates cellulaires sont des structures versatiles et puissantes
dont le formalisme originel impose néanmoins quelques limitations;
l'une des principales étant, à nos yeux, un maillage régulier et
statique. Pour répondre à notre problématique, il est nécessaire
d'employer une structure respectant les critères suivants :

\begin{itemize}
\item{Elle doit partitionner l'espace, possiblement de façon
  irrégulière;}
\item{Des relations de voisinages pourront être déduites de sa
  topologie;}
\item{Elle doit pouvoir représenter à la fois la parcellisation du
  territoire et le réseau routier.}
\end{itemize}

Le diagramme de Voronoï est un candidat idéal. Sa constitution est
intrinsèquement spatiale puisqu'il s'agit d'un partionnement axé
autour de points spéciaux, les générateurs, chacun possédant une
cellule contenant tous les points plus proches de ce générateur que de
tout autre. Autrement dit, la distance séparant un point $p$ placé
dans une cellule de Voronoï et le générateur de cette même cellule est
inférieure à la distance séparant $p$ de tous les autres générateurs
\cite{Edwards1993}. La figure \ref{fig:voronoi} fournit un exemple de
diagramme de Voronoï et on remarque que, naturellement, deux
générateurs voisins sont équidistants de l'arête les séparant et le
segment les reliant y est perpendiculaire.

\begin{figure}[h]
  \centering
  \includegraphics[width=0.5\linewidth]{images/voronoi.png}
  \caption{Un diagramme de Voronoï. Chaque point noir est un générateur.}
  \label{fig:voronoi}
\end{figure}

Les diagrammes de Voronoï trouvent de nombreuses applications en
science. En robotique, les obstacles présents dans un environnement
peuvent être assimilés à des générateurs et un robot cherchant à
maximiser leur évitement préférera longer les frontières des cellules
(les arêtes de Voronoï) \cite{Garrido2006}. En sociologie
géographique, ils permettent d'opposer les zones d'influence de
différents éléments urbains et répondent à des questions telles que :
quel magasin un piéton sera-t-il plus susceptible de visiter selon la
zone dans laquelle il se trouve ? Leur utilisation pour l'étude de
l'épidémie de choléra londonienne en 1854 à permis de vérifier le lien
entre fontaines publiques infectées (les générateurs) et zones
souffrant d'un fort taux de mortalité (les cellules)
\cite{Thomas2010}.

Comme leur homonymie le laisse supposer, la cellule de Voronoï
remplace la cellule carrée de l'automate cellulaire. On remarque
qu'une grille régulière, comme celles présentes dans les automates
cellulaires classiques correspond à un diagramme de Voronoï dans
laquelle les générateurs sont alignés et régulièrement disposés. Une
tesselation de Voronoï peut être considérée comme une généralisation
de la structure grillagée ; notre première contrainte est satisfaite.

À l'échelle de ce modèle, chaque cellule représente une parcelle
cadastrale et on utilise comme générateur le centre de son
empreinte. Le diagramme permet d'identifier les parcelles voisines
comme étant celles partageant une arête de Voronoï. Un graphe de
voisinage est ainsi construit, et adopte la forme duale du diagramme
de Voronoï : la triangulation de Delaunay. Ce premier graphe décrit le
réseau de voisinage mettant en relation les parcelles en contact à
partir du diagramme et satisfait donc la seconde contrainte.

Cette structure permet de décrire un canevas urbain de base dans
lequel l'espace d'influence de chaque parcelle est décrit mais la
composante routière reste encore absente du modèle. Chaque arête de
Voronoï indique un espace entre deux parcelles et est donc susceptible
d'accueillir une route. Dans une véritable ville, chaque parcelle
n'est pas encerclée de voies et l'un des objectifs de la simulation
est de déterminer quelles arêtes accueilleront des routes. Le
diagramme de Voronoï suffit bien à représenter à la fois les éléments
du viaire et du bâti et notre dernière contrainte est comblée.

En réalité, dans ce modèle la ville est représentée par deux graphes
et le diagramme de Voronoï est uniquement employé en tant que point de
départ. Le premier, le graphe du bâti, a pour n\oe ud les centres des
parcelles alors que ses arêtes symbolisent les relations de
voisinage. Le second, le graphe viaire, a des arêtes représentant les
routes et des n\oe uds carrefour joignant plusieurs voies. Les
structures des graphe viaire et bâti sont donc entièrement fondées sur
le diagramme de Voronoï puisqu'il s'agit, respectivement, de
l'ensemble des arêtes et sommets de Voronoï et de sa triangulation de
Delaunay. REFORMULER

Il est essentiel de dissocier le polygone convexe qu'est la cellule de
Voronoï associée à une parcelle et la véritable empreinte cadastrale
de cette dernière. Une cellule représente l'influence d'une parcelle
dans l'espace urbain et possède comme seul point commun avec
l'empreinte son centre puisqu'il s'agit du générateur de la
cellule. Similairement, une arête peut indiquer qu'une voie passe
entre deux parcelles sans pour autant fournir ses coordonnées ou sa
courbure. Si l'on souhaite, dans un but infographique, générer une
image de notre ville à partir de ce modèle, un travail
d'interprétation est nécessaire et n'a pas été traîté à l'occasion de
ce projet. Un exemple est visible sur la figure \ref{fig:interp}.

\begin{figure}[h]

  \centering
  \subcaptionbox{}[0.9\linewidth][c]{
    \includegraphics[width=.3\linewidth]{images/voronoi-interp0.png}
  }

  \subcaptionbox{}[.3\linewidth][c]{
    \includegraphics[width=.3\linewidth]{images/voronoi-interp1.png}
  }
  \subcaptionbox{}[.3\linewidth][c]{
    \includegraphics[width=.3\linewidth]{images/voronoi-interp2.png}
  }
  \subcaptionbox{}[.3\linewidth][c]{
    \includegraphics[width=.3\linewidth]{images/voronoi-interp3.png}
  }

  \caption{Un diagramme de Voronoï trivial et trois interprétations possibles.}
  \label{fig:interp}
\end{figure}

\subsection{Potentialité}

Via le terme \textit{potentialité}, on souhaite exprimer l'opposition
entre deux types d'éléments : les \textit{potentiels} et les
\textit{construits}.

Un élément \textit{construit} est une parcelle ou une voie dont
l'existence physique est avérée. Il existe \textit{en dur} et affecte
ses alentours. L'ensemble des éléments construit forme la ville (voir
figure \ref{fig:construit}).

\begin{figure}
  \centering
  IMAGE
  \caption{Les éléments construits forment la ville.}
  \label{fig:construit}
\end{figure}

Un élément \textit{potentiel} peut être assimilé à une idée germant
dans l'esprit de l'urbaniste ; à une possibilité envisagée. Un élément
potentiel est par la suite soit construit, soit ignoré et oublié. Il
sert de prévision à court-terme quant à l'avenir de la ville et guide
sa morphogénèse. La figure \ref{fig:potentiel} reprend la micro-ville
de la figure \ref{fig:construit} et laisse apparaître voies et
parcelles potentielles.

\begin{figure}
  \centering
  IMAGE
  \caption{Les éléments potentiels guident la croissance
    de la ville.}
  \label{fig:potentiel}
\end{figure}

Un élément potentiel, n'étant pas actif au sein de la ville et
appartenant uniquement au domaine du prévisionnel, n'a pas d'influence
sur les éléments construits. Par contre, la construction de nouveaux
éléments peut en dépendre \textit{e.g.} une route peut être construite
en conséquence à cette prévision, comme attirée par cette portentielle
future installation. Cette dualité dont les relations d'influence sont
clairement unidirectionnelles est inspirée du cycle réel
d'urbanisation que l'on pourrait grossièrement décomposer en ces
quelques étapes :

\begin{enumerate}
\item{Un urbaniste prévoit une nouvelle installation en bordure de
  ville}
\item{Cette prévision attire la route}
\item{La nouvelle route et l'installation potentielle attirent
  d'autres installations potentielles}
\end{enumerate}

REFORMULER PLUS CLAIREMENT

L'essence du modèle est de placer des éléments potentiels en fonctions
de qualités internes au système puis de choisir lesquels véritablement
construire. Ce travail a été décomposé en quatre mécanismes distincts.

\subsection{Mécanismes}

\subsubsection{Automate cellulaire graphe}

La dynamique de croissance urbaine est décomposable sur deux axes. La
croissance horizontale décrit l'expansion spatiale de la ville dont
l'enveloppe grandit pour occuper plus de territoire tandis que la
croissance verticale correspond à l'augmentation des densités au sein
de la ville, souvent à partir d'un ou de plusieurs centres. Le
mécanisme cellulaire présenté ci-après émule la croissance verticale
et les variations de densité internes au système.

On voit la ville est un groupement d'installations humaines axé autour
d'un ou de plusieurs centres. Puisqu'il s'agit avant tout d'un foyer
de population, les choix appliqués par les autorités sont motivés par
un besoin de la rendre toujours plus fonctionnelle. Cela se traduit,
par exemple, par un réseau routier construit dans une optique
d'optimalité quant au déplacement des habitants, même si la réalité
est peuplée de contraintes empêchant une organisation parfaite. Au
niveau parcellaire, cela se traduit par un désir d'équilibrer les
paysages urbains et d'éviter les contrastes forts. La population et
plus précisément sa densité sont donc des facteurs majeurs à prendre
en compte, si ce ne sont les principaux. La ville évolue, de nouveaux
bâtiments apparaissent, d'autres sont rasés, les quartiers changent et
le modèle doit être capable de simuler ces changements. C'est bien sûr
par un automate cellulaire, quoi que quelque peu relaxé, que cet
aspect est géré.

On propose de représenter trois différents types de densité
\textit{faible}, \textit{moyenne} et \textit{élevée} bénéficiant
d'attractivité variantes de façon à ce que les parcelles faibles
côtoient les parcelles moyennes, les parcelles moyennes côtoient les
parcelles élevées et les parcelles faibles s'éloignent des parcelles
élevées. Ce processus d'attraction/répulsion évoque le modèle de
ségrégation de Schelling à la différence qu'ici trois types de
\textit{population} interagissent et qu'il n'y a pas de contrainte de
démenagement (si une parcelle \textit{faible} disparaît, elle ne doit
pas nécessairement réapparaître ailleurs). Ainsi un dégradé discret de
densité apparaît comme dans une ville réelle. La problématique étant
d'étudier la coévolution de deux aspects urbains et non seulement
l'évolution des densités (qui ne sert que de support à l'essort de la
ville), on a préféré choisir une règle basique. Malgré cette
simplicité, le modèle est conçu pour que chaque mécanisme puisse être
modifié indépendamment des autres et il est tout à fait possible
d'utiliser par la suite un automate cellulaire plus sophistiqué pour
améliorer le réalisme de la simulation. On commence par appliquer
cette règle a un automate cellulaire classique (voir figure
\ref{fig:ac}).

\[
\begin{pmatrix}
  1 & 1 & 1 \\
    & 1 & 1 \\
    &   & 1
\end{pmatrix}\]


\begin{figure}[h]
  \centering
  \subcaptionbox{}[.3\linewidth][c]{
    \includegraphics[width=.3\linewidth]{images/logo-litis.png}
  }
  \subcaptionbox{}[.3\linewidth][c]{
    \includegraphics[width=.3\linewidth]{images/logo-litis.png}
  }
  \subcaptionbox{}[.3\linewidth][c]{
    \includegraphics[width=.3\linewidth]{images/logo-litis.png}
  }

  \subcaptionbox{}[.3\linewidth][c]{
    \includegraphics[width=.3\linewidth]{images/logo-litis.png}
  }
  \subcaptionbox{}[.3\linewidth][c]{
    \includegraphics[width=.3\linewidth]{images/logo-litis.png}
  }
  \subcaptionbox{}[.3\linewidth][c]{
    \includegraphics[width=.3\linewidth]{images/logo-litis.png}
  }
  \caption{Trois configurations successives de l'automate
    cellulaire. On y retrouve peu de similarités.}
  \label{fig:ac}
\end{figure}

On remarque que chaque itération transforme entièrement l'état de
l'automate. Cette situation pourrait être acceptable pour d'autres
types d'application mais certainement pas pour une simulation urbaine
dans laquelle les changements sont censés s'opérer à l'échelle de la
décennie et une relative continuité est attendue \textit{i. e.} un
bâtiment est construit pour durer, on ne le remplace pas chaque mois !
Il est donc important d'associer à chaque cellule un élan favorisant
la persistance de son état selon son âge. Les fonctions sigmoïdes,
fréquemment employées en modélisation de systèmes complexes, sont
idéales pour exprimer cette probabilité de changement en fonction du
temps. DETAILS

\begin{figure}[ht]
  \centering
  \subcaptionbox{}[.4\linewidth][c]{
    $f(x) = \frac{1}{1 + e^{-x}}$
  }
  \subcaptionbox{}[.4\linewidth][c]{
    \includegraphics[width=.3\linewidth]{images/logo-litis.png}
  }
  \caption{Sigmoïde classique.}
  \label{fig:sigmoide1}
\end{figure}

\begin{figure}[ht]
  \centering
  \subcaptionbox{}[.4\linewidth][c]{
    $f(t) = \frac{1}{1 + e^{-0.1t-50}}$
  }
  \subcaptionbox{}[.4\linewidth][c]{
    \includegraphics[width=.3\linewidth]{images/logo-litis.png}
  }
  \caption{Probabilité de changement d'état en fonction du temps.}
  \label{fig:sigmoide1}
\end{figure}

On peut voir sur la figure \ref{fig:ac-stable} que le système est plus
stable et que ETC

FAIRE IMAGE : T0 T1 T2 T10 T50 T100

\begin{figure}[h]
  \centering
  \subcaptionbox{}[.3\linewidth][c]{
    \includegraphics[width=.3\linewidth]{images/logo-litis.png}
  }
  \subcaptionbox{}[.3\linewidth][c]{
    \includegraphics[width=.3\linewidth]{images/logo-litis.png}
  }
  \subcaptionbox{}[.3\linewidth][c]{
    \includegraphics[width=.3\linewidth]{images/logo-litis.png}
  }

  \subcaptionbox{}[.3\linewidth][c]{
    \includegraphics[width=.3\linewidth]{images/logo-litis.png}
  }
  \subcaptionbox{}[.3\linewidth][c]{
    \includegraphics[width=.3\linewidth]{images/logo-litis.png}
  }
  \subcaptionbox{}[.3\linewidth][c]{
    \includegraphics[width=.3\linewidth]{images/logo-litis.png}
  }
  \caption{Trois configurations successives de l'automate cellulaire
    stabilisé.}
  \label{fig:ac-stable}
\end{figure}

Les exemples précédents permettent d'illustrer les règles de
transition et met en évidence le problème de stabilité mais le
principe même de cet exposé est de se détacher de la régularité
spatiale contraignante des automates cellulaires et c'est à cet effet
que l'on a présenté le diagramme de Voronoï. À la manière des
automates cellulaires graphes de O'Sullivan \cite{O'Sullivan2000},
chaque parcelle verra son état varier en fonction de son
voisinage. Voisine établit non pas par ??? mais par la topologie du
diagramme de Voronoï issu des positions des centres des parcelles. Il
est à noter qu'à la différence de O'Sullivan, la couverture de
l'espace est totale puisque l'on ne représente pas les parcelles
exactes mais leur cellule de Voronoï.

Le \textit{graphe du bâti} prend ici la forme duale du diagramme de
Voronoï : la triangulation de Delaunay et décrit les relations de
voisinage. Un n\oe ud correspond au centre d'une parcelle et une arête
lie deux parcelles en tant que voisins.

IMAGE

Lors de l'éxécution de ce mécanisme cellulaire, les parcelles
construites influencent tous leur voisins tandis que les parcelles
potentielles ??? unidirectionnel (BONNE QUESTION).

TRIPLE IMAGE

\subsubsection{Placement des parcelles potentielles}

Le second mécanisme place de nouvelles parcelles en bordure de la
ville et est responsable de sa croissance dans l'espace. Les nouvelles
parcelles placées bénéficient néanmoins d'un statut spécial car elles
sont considérées comme \textit{potentielles} et se démarquent des
parcelles \textit{construites} par leur impact sur l'évolution du
système :

\begin{itemize}
\item{L'état d'une parcelle potentielle dépend de toutes ses voisines}
\item{Une parcelle potentielle n'influence pas l'état des parcelles
  voisines construites}
\item{Une parcelle potentielle peut accueillir une route}
\end{itemize}

Cette influence unidirectionnelle dans le mécanisme cellulaire permet
à la parcelle potentielle d'être prête et accordée à son environnement
proche si elle est construite par la suite sans pour autant que les
parcelles déjà construites ne soient influencées par une parcelle
n'existant pas.

En quelques mots, le placement se déroule comme suit :

\begin{enumerate}
\item{On détermine les centres de la ville en fonction de la densité}
\item{On dépose sur un des centres une \textit{graine} mobile qui
  servira de générateur à la nouvelle parcelle}
\item{La graine se déplace sous l'influence des variables inhérentes à
  la ville}
\item{Quand la graine stoppe son mouvement, on y crée la parcelle}
\end{enumerate}

DETECTION DES CENTRES

Le déplacement de la graine est un processus pouvant potentiellement
prendre en compte de nombreuses variables. Dans l'application
d'exemple que l'on décrit, seules la densité et le placement des
routes peuvent guider la graine, car ce sont les seuls données
considérées. Cependant, on souhaite que le modèle soit extensible et
qu'il soit capable de supporter d'autres variables et contraintes : la
valeur des sols par exemple, ou bien la pente des zones envisagées ou
l'impossibilité de s'installer sur certains types de terrain (forêts,
plans d'eau). De cette idée de graine se déplaçant en fonction
d'influences diverses transpire un véritable aspect physique.

Pour rester en accord avec cet aspect physique on emploie un champ de
vecteur généré à partir de l'état du sytème urbain pour guider la
graine vers sa destination. Puisque de nombreux paramètres sont à
prendre en comptre, on utilise un champ de vecteur par paramètre que
l'on souhaite exprimer puis on les combine, ce qui permet de pondérer
l'impact de chaque donnée.

Par exemple,

DENSITE

PATTERN

COMMENTAIRE

IMAGES : densité / routes / obstacle / altitude

ARRET (AIRE ? SIGMOIDE ?)

\subsubsection{Construction des parcelles}

CHOIX

TAUX DE CROISSANCE

\subsubsection{Expansion du réseau routier}

Le placement des parcelles se base sur le réseau routier puisque pour
qu'une parcelle soit construite, elle doit être reliée à au moins une
voie. Les voies sont quant à elle dépendantes des parcelles car selon
notre définition, une route est une arête de Voronoï d'une parcelle
(potentielle ou construite). Ce cycle est similaire à la morphogénèse
d'une ville car le processus d'extension que l'on observe dans un cas
d'urbanisme réel est le suivant :

\begin{enumerate}
\item{On prévoit d'installer des bâtiments en bordure de ville (les
  parcelles potentielles)}
\item{On construit une route avant de construire les parcelles}
\item{On construit plus tard les parcelles}
\end{enumerate}

DEJA DIT AILLEURS

les routes sont donc supports aux parcelles et leur construction est
motivée par la perspective de nouvelles parcelles. Les parcelles
potentielles étant placées par un mécanisme explicité plus haut, il
reste à sélectionner quelles arêtes du diagramme de Voronoï vont
devenir de véritables routes. On retrouve ici aussi le vocabulaire de
la potentialité utilisé au niveau parcellaire : les arêtes de Voronoï
sont considérées comme routes \textit{potentielles} et

La sélection les routes aptes à être construites doit être justifiée
par un besoin fonctionnel assurant une activité interne de la ville
efficace. Dans le cas du réseau viaire, on cherche donc à éviter les
engorgements et donc à organiser le maillage routier de façon à ce que
les zones à forte densité soient correctement desservies. DENSITE

NETWORK SIMPLEX

CONNEXITE

\section{Construction}

Notre système urbain est décrit sur deux niveaux; plus précisément,
par deux graphes. On distingue le graphe parcellaire du graphe viaire
car, bien qu'ils soient étroitement liés, ils représentent des couches
différentes du réseau urbain.

QQUAD MACHIN NON, 2 GRAPHES OUI

\subsection{Le graphe parcellaire}

Le graphe parcellaire correspond à la couche bâti. Ses n\oe uds sont
les centres des parcelles et chacune de ses arêtes représente une
relation de voisinage entre deux parcelles. Un diagramme de Voronoï
est nécessaire à la construction de ce graphe. Les seules données dont
l'on a besoin en entrée sont donc les positions centrales des
parcelles. Ces coordonnées peuvent être extraites de fichiers
d'informations géographiques mais dans un premier temps, et dans un
but llustratif, on se cantonne à utiliser des positions aléatoirement
choisies.

\begin{figure}
  \centering
  \includegraphics[width=.6\linewidth]{images/logo-litis.png}
  \caption{}
  \label{fig:construction-bati1}
\end{figure}

La librairie Java JTS \cite{JTS} est spécialisée dans les traitements
géométriques et permet notamment de générer un diagramme de Voronoï à
partir d'une liste de points. Le résultat d'un tel traitement prend la
forme d'un ensemble de polygones, chacun représentant une cellule de
Voronoï, mais aucune autre information, notamment d'adjacence, n'est
fournie. Une fois le graphe parcellaire peuplé par des n\oe uds
positionnés aux coordonnées fournies plus tôt, il reste donc à
calculer les relations de voisinage.

\begin{figure}
  \centering
  \includegraphics[width=.6\linewidth]{images/logo-litis.png}
  \caption{}
  \label{fig:construction-bati2}
\end{figure}

Puisque l'on ne dispose que des polygones formant le diagramme de
Voronoï et des positions de leur centroïdes, on procède par
l'utilisation de tests géométriques. Naturellement, si deux cellules
sont en contact -- si elle partagent une arête -- alors on relie les
deux n\oe uds du graphe leur étant associés.

\begin{figure}
  \centering
  \includegraphics[width=.6\linewidth]{images/logo-litis.png}
  \caption{}
  \label{fig:construction-bati3}
\end{figure}

On observe sur la figure \ref{fig:construction3} que les parcelles
situées en bordure de la ville sont dotées d'immenses cellules de
Voronoï. Afin de ne pas influencer négativement le déroulement de la
simulation, dans laquelle l'aire des cellules peut avoir un impact sur
l'évolution du système, on préfère modifier les contours du diagramme
pour qu'il épouse la forme générale de la ville. Pour ce faire, on
calcule l'enveloppe convexe du jeu de points utilisés comme
coordonnées initiales, on l'agrandit de quelques unités afin que les
parcelles ne soient pas collées à la bordure extérieure et on conserve
uniquement l'intersection du diagramme original et de cette enveloppe.

EN FAIT NON, GARDER ?

La finalité de cet exercice n'est pas seulement de faire évoluer
l'automate cellulaire irrégulier que forme cette structure mais aussi
de lui permettre de se transformer au cours du temps, de s'étendre par
morphogenèse. INSERTION, DELETION

\subsection{Le graphe viaire}

Le graphe viaire correspond à la couche routière. Ses n\oe uds sont
des carrefours, aux croisements des parcelles, et ses arête des
voies. En pratique, pour différencier les routes potentielles des
routes construites, on attribue une étiquette spéciale à ces
dernières.

Comme lors de la construction du graphe parcellaire, on commence par
placer les n\oe uds (ici des carrefours) puis on veut les relier par
des arêtes (les routes).

Dans un premier temps, il nous faut déterminer les positions des
croisements qui feront office de n\oe uds dans le graphe viaire. Cette
phase est basée sur l'analyse du graphe parcellaire construit plus
tôt. Puisqu'une route entre deux bâtiments peut être définie par une
arête de Voronoï partagée entre deux parcelles, il est logique
d'admettre qu'un croisement correpond à un sommet de Voronoï partagée
par deux cellules ou plus. Le but de l'étape décrite est donc de
déterminer des groupes de cellules axées autour d'un même sommet
pivot.

\begin{figure}
  \centering
  \includegraphics[width=.6\linewidth]{images/logo-litis.png}
  \caption{}
  \label{fig:construction-viaire1}
\end{figure}

On remarque que ces clusters de parcelles semblent être des
sous-graphes complets maximaux. Dans un premier temps, l'utilisation
de l'algorithme de Bron-Kerbosch a été envisagé (et appliqué,
inutilement !) mais certains cas particuliers nous laisse entrevoir
le fait qu'une contrainte supplémentaire est manquante. En effet, la
recherche d'une clique maximale par l'algorithme cité ne prend pas en
compte la nécessité que les cellules du cluster partagent un
sommet. Sur l'exemple de la figure ???, quatres cellules forment une
clique sans pour autant avoir un sommet de Voronoï en commun.

\begin{figure}
  \centering
  \includegraphics[width=.6\linewidth]{images/logo-litis.png}
  \caption{}
  \label{fig:construction-viaire2}
\end{figure}

Puisque que l'algorithme de Bron-Kerbosch se révèle inadapté à l'usage
que l'on souhaitait en faire, il a été nécessaire de réflechir à une
autre méthode pour déterminer la position des carrefours et, surtout,
les parcelles à y associer. Une procédure a donc été mise en place
pour détecter ces clusters de parcelles pivotant autout d'un même
croisement. L'idée de base est de construire tous les groupes de
parcelles tels que :

\begin{itemize}
\item{Toutes les parcelles d'un groupe soient voisines entres elles}
\item{Les groupes soient maximaux}
\item{Les parcelles d'un groupe aient tous un sommet en commun}
\end{itemize}

Le dernier critère est la contrainte manquante à la définition des
cliques maximales. Concrètement, la construction de toutes ces listes
se résume à la construction de tous les cycles BLABLA. Ensuite,
ROUTES.

Un désavantage de cette méthode, outre sa complexité algorithmique,
est que seules les arêtes partagées par au moins deux parcelles
deviennent des routes. Ainsi, on remarque que les arête délimitant la
bordure de la ville (celles à l'extérieur du diagramme de Voronoï)
sont absentes. Il semble pourtant nécessaire que toutes les arête du
diagramme deviennent de potentielles voies car, dans notre modèle, la
route est le support du bâti. Et si aucune route ne peut se construire
à la bordure de la ville, aucun bâtiment ne s'y installera et la ville
ne grandira jamais.

Une solution MAIS

\begin{figure}
  \centering
  \includegraphics[width=.6\linewidth]{images/logo-litis.png}
  \caption{}
  \label{fig:construction-viaire3}
\end{figure}

Finalement, nous nous sommes tournés vers une solution plus claire,
plus simple et permettant de transfomer en routes potentielles toutes
les arêtes sans exception : la construction puis la fusion de
sous-réseaux routiers. Dans le cadre de la tentative précédente,
chaque carrefour était identifié est placé dans le graphe viaire sous
la forme d'un n\oe ud mais la phase de placement des routes (la
liaison des n\oe uds par des arêtes) se faisaient une fois tous les
croisements placés. Il est clairement plus simple de créer pour chaque
parcelle un sous-réseau routier composé de ses propres sommets et de
els relier immédiatement. Une fois tous ces petits graphes crées ont
les fusionnent tous en prenant comme critère d'identification des
sommets leur position. On obtient au final un réseau routier complet
duquel aucune arête ne manque.

IMAGE

\section{Mesures}

HEUUUUUUUU...

DEGRE DES CARREFOURS

ELOIGNEMENT DE LA DENSITE PAR RAPPORT AU CENTRE GEOMETRIQUE

ELOIGNEMENT DE LA DENSITE PAR RAPPORT AUX CENTRES

TAILLE DES PARCELLES ? (BIAIS AU BORD)

\section{Conclusion}

RESUME

CONSTAT

PERSPECTIVES (DYNAMIQUES INTERNES, REALISME, INTERPRETATION, VRAIE VILLE)

\printbibliography

\end{document}
