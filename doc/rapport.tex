\documentclass[12pt]{article}

\usepackage[utf8]{inputenc}
\usepackage[francais]{babel}

\usepackage{amsmath}
\everymath{\displaystyle}

\usepackage{caption}
\usepackage{subcaption}
\usepackage{float}

\usepackage{tikz}
\usetikzlibrary{calc}

\definecolor{densitylow}{RGB}{255,220,220}
\definecolor{densitymedium}{RGB}{255,130,130}
\definecolor{densityhigh}{RGB}{255,50,50}

\usepackage[natbib=true,sorting=none,style=numeric,url=false,doi=false,isbn=false]{biblatex}
%\addbibresource{references}
\bibliography{references}

\begin{document}

\begin{titlepage}

  Stage de recherche

  {\Large \textsc{Coévolution du réseau viaire et du bâti}}

  \begin{flushright}
    \textit{Auteur :}\\
    Merwan {\scshape Achibet}\\[0.5cm]
    \textit{Encadrants :}\\
    Stefan {\scshape Balev}\\
    Antoine {\scshape Dutot}\\
    Damien {\scshape Olivier}
  \end{flushright}

  \vfill

  \begin{center}
    ILLUSTRATION
  \end{center}

  \vfill

  \begin{center}
    \includegraphics[width=.25\linewidth]{images/logo-univ-le-havre.png}
    \qquad\qquad\qquad
    \includegraphics[width=.25\linewidth]{images/logo-litis.png}
  \end{center}

  \begin{center}
    {\small Mars - Juin 2012}
  \end{center}

\end{titlepage}

\begin{center}
  {\scshape\textbf Keywords}
\end{center}

{Urban system, city morphogenesis, Voronoi diagram, cellular automaton.}

\begin{center}
  {\scshape\textbf Extended abstract}
\end{center}

Gathering issues of human, economic, geographic and political nature,
the city truly is a complex system. The increasing growth in
population produces urban systems like the world has never seen
before, of increasing size, increasing heterogeneity and increasing
complexity. Studying the relationships between its internal elements
is fundamental to the understanding of its mechanics and to better
predict their developement. This work focuses on the relationship
between the pattern of human installations within the city --
characterized at the atomic level by a basic subdivision, the land lot
-- and its road network.

Systematic definitions of the city are many. Through the
anthropologist's eye it can be seen as a concentration of persons, the
economist will prefer to view it as a support for the exchange of
financial and physical assets and the urbanist as a functional entity
composed of flows and services. We choose to consider that the
evolution of a city is driven by its population, translated as a
measure of its density. This goes in hand with our focus as land lots
are inhabited by the same people that uses the surrounding roads to go
from one place to another and urbanistic decisions are motivated by a
need to optimize the city, streamlining transport means and avoiding
high contrasts in the urban fabric.

Study of the domain of urban simulations shows that available
scientific works can be classified in two different categories. For
visual rendering purposes, special effects in movies or video games,
methods have been proposed that generate visually satisfying cities
without regarding realism as an obligation. These are often based on
empirical observations and the main idea is to emulate the street
patterns found in any urban system. Scientific modeling, on the other
hand, prefers to focus on the inner qualities of a city, often by
studying a subset of those, to analyze its current state and
extrapolate its future. We orient our research with the latter in mind
but the former represents a non-negligible source of inspiration.

A reader browsing through the field litterature will often encounter
cellular automata. These structures, which applicability and potential
complexity has been proven over times, are fit for describing any kind
of space-related problem. Nevertheless, their rigorous formalism may
sometimes restrain realism and negatively impact the validity of the
model. For example, a cellular automaton topology is, by definition,
perfectly regular and representing a city with a set of identical and
aligned cells seems like a coarse simplification. In the same way, the
fact that each cell has an identical neighborhood structure, the
temporal synchronism and the state discretization may be
questionned. We take a drastic step towards realism and embrace the
spatial aspect of the city by replacing the classic cellular automaton
with a Voronoï diagram following the same basic concepts. Each of its
cell represents a land lot and neighborhood relationships that are
ruled by adjacency determine its future state ; the regularity
constraint is thus relaxed. Voronoi edges delineate the Voronoi space
of land lots and, as such, are perfect supports for roads.

The different elements represented in our model (\textit{i. e.} land
lots and roads) are declined in two flavors. \textit{Potential}
elements have an ethereal status ; their spatial characteristics are
known but until they are definitely built, they have no direct effect
on the overall city and only represent an idea, a possible
outcome. \textit{Built} elements were potential elements that have
been constructed. They form the physical city. Concisely, the gist of
the model is to add potential elements to the city and, only later,
chose which ones will be built and which ones will be forgotten. The
road network expands to accomodate the growth in density and to
support anticipated land lots. This process has been divided into
three separate mechanisms.

The cellular part of the model lets the inner variables of the city
vary based on the Voronoi tesselation and a set of simple rules. Here,
only population density is considered and, as a result, the
characteristic gradual patterns found in most of the towns of the
world is reproduced.

Whereas the previous part of the model is ensuring vertical growth,
the horizontal component of the evolution of the city is handled by an
extensible method based on the use of vector fields. Each piece of
information that we want to consider is represented by a vector field
that will guide the placement of new lots. For example, a field
escaping from high density areas is used to ensure urban
sprawl. Another makes new lots move towards the closest roads such
that they remain snapped to the main transport axes. All vector fields
are then summed up with varying coefficient. This general approach can
model any kind of guidance or constraint ; in particular obstacle
avoidance so that the city does not extends itself on forbidden areas
like lakes, beaches or protected forests.

The third and last mechanism chooses which of the potential elements
are to be permanently built. Potential roads are chosen with respect
to their contribution to the global network, by means of a network
flow evaluation. Potential land lots are chosen depending on their
position relative to the built roads and the date of their addition
into the potential domain.

RESULTATS/MESURES

\newpage

\tableofcontents

\newpage

REMERCIEMENTS

\newpage

\section{Introduction}

LES VILLES GRANDISSENT

STATISTIQUES

Le système complexe que forme la ville est donc soumis à une forte
croissance et un réel besoin de contrôler et de prévoir son évolution
se fait ressentir suite à l'explosion démographique que subissent les
zones urbaines. C'est dans ce cadre de contrôle et de prévision qu'il
est nécéssaire d'étudier les mécanismes sous-jacents à une ville pour
pouvoir les reproduire sous forme de simulations et étudier différents
scénarios possibles.

On se concentre ici sur deux domaines majeurs du tissu urbain, le
viaire et le bâti, afin d'étudier leur relation de coévolution. Ces
deux aspects de la ville semblent aller de paire puisque le bâti
abrite les populations tandis que le viaire leur permet de se déplacer
d'un point à un autre. C'est la répartition des parcelles et la
structure du réseau routier existant qui guide l'expansion d'une
ville. Il est ainsi naturel d'utiliser la densité de population comme
force de guidage à son évolution.

ETOFFER

La première partie présente un état de l'art de la modélisation de
systèmes urbains et se concentre particulièrement sur les méthodes à
base d'automates cellulaires tout en adoptant un point de vue
historique pour justifier l'adoption de cette structure. Le modèle
conçu dans le cadre de ce stage de recherche est ensuite présenté en
seconde partie. Enfin, des mesures diverses sont employées pour tester
et valider ce travail dans la troisième partie.

\section{\'Etat de l'art}

\subsection{Automates cellulaires et simulation urbaine}

La modélisation de systèmes complexes est longtemps uniquement passée
par l'usage de méthodes mathématiques; typiquement, des systèmes
d'équations différentielles. Ces techniques permettent de décrire des
lois d'évolution et d'observer, ainsi que de prédire par
extrapolation, le comportement de phénomènes du réel. Dans le cas de
modèles prenant en compte un vaste jeu de paramètres, cette approche
peut néanmoins se révéler délicate à employer. Plus intrinsèquement,
même si une telle modélisation est basée sur des observations ancrées
dans la réalité, il s'agit d'une représentation conceptuelle d'un
problème et aucune mimique des mécaniques sous-jacentes ne s'opère.

Historiquement, les prémices de l'informatique moderne et d'un tout
autre paradigme de modélisation sont à attribuer aux esprits du milieu
du vingtième siècle. Alan Turing introduit en 1936 la machine éponyme
qui, bien que purement théorique, possède un module de contrôle ainsi
qu'une mémoire et peut donc exécuter une infinité d'algorithmes. Cette
démarche se démarque de l'approche mathématique et semble plus
humaine; on ne résout pas un problème en utilisant des fonctions
associant une quantité à un résultat mais on agit véritablement sur
ses données. L'idée de base de Turing était d'ailleurs d'assimiler le
fonctionnement de sa machine au travail d'une personne remplissant les
cases d'un tableau infini.

Entraîné par cette mouvance procédurale et en réaction aux réseaux de
neurones de McCulloch et Pitts, John von Neumann et Stanislaw Ulam
joignent leurs travaux durant les années 40 pour concevoir l'automate
cellulaire : un système comprenant un ensemble d'automates à états
spatialement localisés (typiquement sous forme de grille) et
interconnectés en fonction de leur proximité. Les entrées de chaque
automate correspondent alors aux états des automates voisins et de
cette organisation se dégagent de fortes relations
d'interdépendance. Le jeu de la vie de Conway en est un exemple
classique. La simplicité de ses règles, mise en contraste avec la
variété des configurations engendrées, témoigne de la richesse des
automates cellulaires \cite{Gardner1970}.

Les automates cellulaires ont depuis été extensivement étudiés et sont
appliqués à l'étude de nombreux phénomènes biologiques, physiques et
sociaux \cite{Ganguly2003}. La motivation d'Ulam lors de leur
conception était d'ailleurs de modéliser la croissance de cristaux. On
peut aussi citer en exemple la simulation de la dynamique de fluides
\cite{Frisch1986} et de la croissance de tumeurs
\cite{Kansal2000}. Leur caractère spatial laisse supposer qu'ils sont
particulièrement adaptés aux applications géographiques, et dans le
cadre de notre problématique, urbaines. Ils ne fûrent paradoxalement
pas immédiatement exploités à cet effet et c'est seulement suite à un
article de Waldo Tobler, en 1975, que le rapprochement entre les
automates cellulaires et le domaine de la géographie apparaît
clairement \cite{Tobler1975}. Sont ensuite publiés des travaux majeurs
appliquant l'automate cellulaire à des problématiques géographiques
multi-échelles telles que l'évolution d'épidémies \cite{Fu2003} et la
ségrégation de population \cite{Schelling1969} (voir figure
\ref{fig:schelling}).

\begin{figure}[H]
  \centering
  \includegraphics[width=.6\linewidth]{images/schelling.png}
  \caption{Configuration produite par le modèle de Schelling. Chacune
    des deux couleurs représente une population différente. La
    conclusion de ces travaux est qu'un faible degré d'animosité entre
    deux populations suffit à les séparer de manière évidente.}
  \label{fig:schelling}
\end{figure}

Une idée très exploitée dans ce domaine est d'associer un potentiel de
transition à chaque cellule et ce, vers tous les états qu'elles
peuvent adopter. Dans les modèles déterministes, la transition vers
l'état à plus haut potentiel est appliquée tandis que dans les modèles
stochastiques, un tirage aléatoire biaisé est préféré. Le potentiel
d'une cellule à passer à un nouvel état est déterminé en fonction de
paramètres propres au modèle. Peuvent être pris en compte l'élévation
du terrain, la densité de population, la proximité d'axes routiers, la
proximité de centres urbains, l'âge des parcelle, leur valeur; en
fait, toute combinaison d'attributs relatifs à un réseau urbain. Par
exemple, dans une simulation représentant les différents types
d'usage, le passage d'une cellule à l'état \textit{résidentiel}
pourrait dépendre de la proximité des commerces et des routes et de
l'éloignement des zones industrielles. Bien sûr, un nombre élevé de
paramètres à prendre en compte requiert un couplage fin et l'impact de
chaque variable peut être pondéré. Puisque les variations
individuelles de paramètres n'émergent pas de manière transparente à
la surface de la simulation, les modèles urbains basés sur des
automates cellulaires doivent être finement calibrés et leur réalisme
est un défi en soi. Pour contourner ce problème, Yeh et Li prônent
l'usage d'un réseau de neurones pour pondérer chaque paramètre à
partir de l'analyse de données cartographiques historiques
\cite{Yeh2002}.

Il est important de noter que la simplicité du formalisme enveloppant
un automate cellulaire strict s'oppose à la qualité de la simulation,
notamment dans le cadre de modèles spécifiques
\cite{Torrens2001}. Dans ce cas, une prise de liberté quant aux
formalisme originel est autorisée, voire nécessaire, pour obtenir des
résultats satisfaisants \cite{White1998}.

La première limite que le formalisme cellulaire de base impose est la
discrétisation des états que chaque cellule peut adopter. Même si
cette caractéristique fait partie intégrante des particularités qui
confèrent aux automates cellulaires leur simplicité d'usage et
d'analyse, la description de quantités pouvant arborer un éventail
infini de valeurs est alors impossible. Plus concrètement, il est aisé
de catégoriser les cellules d'un espace selon le fait, par exemple,
qu'elles contiennent des installations humaines ou non (état booléen)
\cite{Benguigui2004,Cornu2008} ou de façon plus sophistiquée, en
fonction de leur type d'usage (\textit{résidentiel},
\textit{commercial} et \textit{industriel} \cite{Lechner} et plus
\cite{Dubos-Paillard2003}). Représenter des quantités réelles et des
variations continues est moins aisé. Pour symboliser plus finement la
densité au c\oe ur d'un ensemble urbain, Semboloni utilise par exemple
un automate cellulaire de dimension trois dans lequel plus une pile de
cellules actives est haute et plus la zone représentée est peuplée
\cite{Semboloni2000}. Plus généralement, il est accepté de représenter
l'état d'une cellule par un vecteur contenant des valeurs réelles; des
règles de transitions adaptées et mesurées sont alors à mettre en
place.

L'homogénéité d'un automate cellulaire fait partie intégrante de sa
définition originelle : en mettant de côté l'état qu'elles adoptent,
toutes les cellules sont identiques en forme et en structure de
voisinage. Dans le cadre de notre problématique, cette approche est
limitante car, dans une ville, les parcelles ne sont que rarement
identiques et alignées. Similairement, la notion de voisinage est
clairement à redéfinir. Pour des problèmes classiques, les voisinages
de von Neumann et de Moore sont régulièrement utilisés mais la
relation par contiguité qu'ils décrivent ne convient pas à la
représentation des liens de dépendance à plus grande échelle se
développant dans un système urbain. Le positionnement d'un bâtiment
résidentiel dans une ville se base évidemment sur le voisinage direct
des zones envisagées (on préfère construire une maison dans un
quartier résidentiel) mais il faut aussi prendre en compte les
alentours plus distants (la centrale thermique se trouvant à 500
mètres du site peut poser problème). Une solution possible est
d'étendre les aires des voisinages de von Neumann et de Moore tout en
conservant leur forme caractéristique. La symétrie évidente se
dégageant de telles simplifications va à l'encontre des relations
prenant place au sein d'une ville dont les zones ne sont que très
rarement parfaitement disposées. O'Sullivan a choisi de relaxer cette
contrainte de partionnement spatial régulier pour faire un pas dans la
direction du réalisme \cite{O'Sullivan2000,O'Sullivan2001} :
conventionnellement, une cellule d'automate correspond à un
sous-espace urbain ou bien une parcelle cadastrale mais dans chacun de
ces cas le modèle se base évidemment sur une simplification grossière
de l'espace étudié. Il décide donc de donner à chaque cellule les
mêmes qualités topologiques que les parcelles qu'elles représentent :
même formes, même dimensions, mêmes coordonnées. Une variété de
relations de voisinage sont alors envisageables (par voisinage au sens
urbain, par distance dans un rayon d'influence, par critère de
visibilité). L'éloignement du formalisme cellulaire est drastique car
la structure perd de son homogénéité (chaque cellule est différente),
la couverture de l'espace n'est plus complète (des vides entre les
cellules apparaissent) et le voisinage diffère lui aussi mais chacun
de ces changements ???.

\begin{figure}[H]
  \centering
  \includegraphics[width=.7\linewidth]{images/gca.png}
  \caption{Hoxton, un quartier de Londres, modélisé par l'automate
    cellulaire graphe de David O'Sullivan \cite{O'Sullivan2000}.}
  \label{fig:sullivan}
\end{figure}

\begin{figure}
  \begin{subfigure}[b]{.5\linewidth}
    \centering
    \begin{tikzpicture}
  \draw[step=1,gray] (0,0) grid (3,3);

  \node at (0.5,0.5) {G};
  \node at (1.5,0.5) {H};
  \node at (2.5,0.5) {I};
  \node at (0.5,1.5) {D};
  \node at (1.5,1.5) {E};
  \node at (2.5,1.5) {F};
  \node at (0.5,2.5) {A};
  \node at (1.5,2.5) {B};
  \node at (2.5,2.5) {C};
\end{tikzpicture}

    \caption{Automate cellulaire classique.}
  \end{subfigure}
  \begin{subfigure}[b]{.5\linewidth}
    \centering
    \begin{tikzpicture}
  \draw (0.5,0.5) node[circle,draw] (g) {G};
  \draw (1.5,0.5) node[circle,draw] (h) {H};
  \draw (2.5,0.5) node[circle,draw] (i) {I};
  \draw (0.5,1.5) node[circle,draw] (d) {D};
  \draw (1.5,1.5) node[circle,draw] (e) {E};
  \draw (2.5,1.5) node[circle,draw] (f) {F};
  \draw (0.5,2.5) node[circle,draw] (a) {A};
  \draw (1.5,2.5) node[circle,draw] (b) {B};
  \draw (2.5,2.5) node[circle,draw] (c) {C};

  \draw (a) -- (b);
  \draw (a) -- (e);
  \draw (a) -- (d);
  \draw (b) -- (c);
  \draw (b) -- (f);
  \draw (b) -- (e);
  \draw (b) -- (d);
  \draw (b) -- (f);
  \draw (c) -- (f);
  \draw (d) -- (e);
  \draw (d) -- (h);
  \draw (d) -- (g);
  \draw (e) -- (c);
  \draw (e) -- (f);
  \draw (e) -- (i);
  \draw (e) -- (h);
  \draw (e) -- (g);
  \draw (f) -- (i);
  \draw (f) -- (h);
  \draw (g) -- (h);
  \draw (h) -- (i);
\end{tikzpicture}

    \caption{Automate cellulaire graphe.}
  \end{subfigure}
\end{figure}

Une prise de liberté quant à l'aspect temporel est aussi
envisageable. Un automate cellulaire strict est synchrone,
\textit{i. e.} les changements d'état de toutes les cellules
s'effectuent simultanément. Si le choix était fait de mettre à jour
chaque état de façon asynchrone, le comportement de l'automate en
serait lourdement modifié. Par exemple, les qualités auto-réplicatives
de certaines entités du jeu de la vie ne seraient pas garanties. Il
est pourtant légitime de se questionner sur la validité d'un tel choix
dans une simulation urbaine, premièrement parce qu'une ville est un
système complexe et désordonné, deuxièmement parce les processus qui
s'y déroulent sont réglés sur différentes échelles temporelles.

Bien que les automates cellulaires soient couramment utilisés pour
simuler le traffic routier (dans leur version 1D CITATION ou 2D
\cite{Queloz1996}), ils s'accordent peu avec la construction même d'un
réseau viaire. Dans les simulations cellulaires urbaines, le
positionnement des routes a un impact sur le développement des
cellules puisque le viaire \textit{attire} le bâti mais le réseau est
souvent fourni en entrée et reste fixe. Nous sommes amener à nous
interroger sur la capacité des automates cellulaires à modéliser le
développement routier. Les relations de proximité les caractérisant
sont-elles adaptées à la construction de structures dont l'échelle est
celle de la ville et non plus celle de la parcelle ? REPONSE

\subsection{Approches alternatives}

Les automates cellulaires ne sont pas l'unique moyen de modéliser la
croissance urbaine. Plusieurs simulations existantes sont des systèmes
multi-agent \cite{Lechner2003,Lechner2004}. Dans ces cas, un agent est
assimilé à un promoteur immobilier et peut acheter des terres, les
vendre, les développer ou changer leur type. Les actions qu'il
entreprend sont évaluées en fonction de l'impact sur la ville
(changement de la valeur immobilière, avis de la population) et des
réglementations locales afin d'éviter toute configuration illégale.
Pour la construction du réseau routier, une solution est de mettre en
place, en plus des agents promoteurs, deux types d'agents
traceurs. Les \textit{extenders} parcourent toute la surface du
terrain à la recherche de bâtiments isolés puis tracent une route
jusqu'au réseau urbain. Les \textit{connectors} se déplacent
uniquement sur le réseau viaire et y raccordent les bâtiments non
connectés se trouvant dans leur rayon de détection
\cite{Lechner2003}. Ce genre d'approche quant à la coévolution entre
routes et bâti introduit un défaut : le réseau viaire est construit à
partir du bâti et des bâtiments peuvent rester isolés. PLUS?

D'autres solutions s'éloignant des systèmes complexes et penchant du
côté de la génération procédurale de contenu existent. Souvent, le
domaine d'application de telles méthodes est l'infographie, le cinéma
et le jeu vidéo et l'objectif est alors de construire de manière
automatique une ville visuellement réaliste sans se soucier de son
caractère fonctionnel. Usuellement, l'organisation parcellaire dépend
entièrement du réseau routier car la première étape est souvent de
générer un réseau viaire complet puis de placer le bâti en subdivisant
récursivement les niches vides formées par les voies. Dans Citygen
\cite{Kelly2006b}, un point $p$ de l'espace est aléatoirement choisi
puis on calcule un ensemble de plusieurs routes raccordant $p$ au
réseau routier existant en faisant varier leur déviation angulaire et
un paramètre de bruit; la route finale est celle pour laquelle la
variation d'altitude est la plus faible. CityEngine \cite{Parish2001}
utilise un L-System dont les règles permettent de reproduire les
différents motifs quadrillés, radiaux et organiques que l'on retrouve
dans une ville. La nature récursive des L-Systems permet à ces motifs
de se combiner et d'apparaître à différents niveaux de profondeur
(voir figure \ref{fig:cityengine}). Dans une autre simulation, le
tracé des routes suit les \textit{hyperstreamlines} \cite{Chen2008}
formées par un champ de vecteurs. Ce champ est calculé par combinaison
de plusieurs autres champs de vecteurs, chacun représentant des
contraintes directionnelles particulières telles que les zones
interdites (eau, espaces verts), l'altitude et la densité de
population. Ces techniques sont intrinsèquement géométriques, et comme
précisé plus haut, le résultat est purement visuel, mais elles
représentent une source d'inspiration à ne pas négliger.

\begin{figure}[H]
  \centering
  \includegraphics[width=.8\linewidth]{images/cityengine.png}
  \caption{CityEngine mélange des motifs urbains extraits de cartes de
    Paris et de New York \cite{Parish2001}.}
  \label{fig:cityengine}
\end{figure}

L'un des rares modèles gérant à la fois l'évolution du réseau viaire
et du bâti est présenté par Weber \cite{Weber2009} et n'emploie pas
d'automate cellulaire. Le principe est le suivant : à chaque
agrandissement du réseau urbain, on crée plusieurs routes virtuelles
en suivant des règles géométriques précises (allongement des voies
existantes, limitation du degré des carrefours à 4, l'angle entre
chaque rue tend vers 90 degrés). Parmi les $n$ routes générées, une
seule sera construite. Pour la choisir, le traffic sur ces nouvelles
routes est simulé par des agents piétons et véhicules et l'on
identifie celle qui sera la plus bénéfique au réseau.

\section{Le modèle}

\subsection{Structure}

Les automates cellulaires sont des structures versatiles et puissantes
dont le formalisme originel impose néanmoins quelques limitations;
l'une des principales étant, à nos yeux, un maillage régulier et
statique. Pour répondre à notre problématique, il est nécessaire
d'employer une structure respectant les critères suivants :

\begin{enumerate}
\item{Elle doit partitionner l'espace, possiblement de façon
  irrégulière;}
\item{Des relations de voisinages pourront être déduites de sa
  topologie;}
\item{Elle doit pouvoir représenter à la fois la parcellisation du
  territoire et le réseau routier.}
\end{enumerate}

Le diagramme de Voronoï est un candidat idéal. Sa constitution est
intrinsèquement spatiale puisqu'il s'agit d'un partionnement axé
autour de points spéciaux, les générateurs, chacun possédant une
cellule contenant tous les points plus proches de ce générateur que de
tout autre. Autrement dit, la distance séparant un point $p$ placé
dans une cellule de Voronoï et le générateur de cette même cellule est
inférieure à la distance séparant $p$ de tous les autres générateurs
\cite{Edwards1993}. La figure \ref{fig:voronoi} fournit un exemple de
diagramme de Voronoï et on remarque que, naturellement, deux
générateurs voisins sont équidistants de l'arête les séparant et le
segment les reliant y est perpendiculaire. DEFINITION PLUS FORMELLE ?

\begin{figure}[!ht]
  \centering
  \includegraphics[width=0.7\linewidth]{images/voronoi.png}
  \caption{Un diagramme de Voronoï. Chaque point noir est un générateur.}
  \label{fig:voronoi}
\end{figure}

Les diagrammes de Voronoï trouvent de nombreuses applications en
science. En robotique, les obstacles présents dans un environnement
peuvent être assimilés à des générateurs et un robot cherchant à
maximiser leur évitement préférera longer les frontières des cellules
(les arêtes de Voronoï) \cite{Garrido2006}. En sociologie
géographique, ils permettent d'opposer les zones d'influence de
différents éléments urbains et répondent à des questions telles que :
quel magasin un piéton sera-t-il plus susceptible de visiter selon la
zone dans laquelle il se trouve ? Leur utilisation pour l'étude de
l'épidémie de choléra londonienne en 1854 à permis de vérifier le lien
entre fontaines publiques infectées (les générateurs) et zones
souffrant d'un fort taux de mortalité (les cellules)
\cite{Thomas2010}.

Comme som homonymie le laisse présager, la cellule de Voronoï remplace
la cellule carrée de l'automate cellulaire. On remarque qu'une grille
régulière, comme celles présentes dans les automates cellulaires
classiques correspond à un diagramme de Voronoï dans laquelle les
générateurs sont alignés et régulièrement disposés. Une tesselation de
Voronoï peut être considérée comme une généralisation de la structure
grillagée ; notre première contrainte est satisfaite.

À l'échelle de ce modèle, chaque cellule représente une parcelle
cadastrale et on utilise comme générateur le centre de son
empreinte. Le diagramme permet d'identifier les parcelles voisines
comme étant celles partageant une arête de Voronoï. Un graphe de
voisinage est ainsi construit, et adopte la forme duale du diagramme
de Voronoï : la triangulation de Delaunay. Ce premier graphe décrit le
réseau de voisinage mettant en relation les parcelles en contact à
partir du diagramme et satisfait donc la seconde contrainte.

Cette structure permet de décrire un canevas urbain de base dans
lequel l'espace d'influence de chaque parcelle est décrit mais la
composante routière reste encore absente du modèle. Chaque arête de
Voronoï indique un espace entre deux parcelles et est donc susceptible
d'accueillir une route. Dans une véritable ville, chaque parcelle
n'est pas encerclée de voies et l'un des objectifs de la simulation
est de déterminer quelles arêtes accueilleront des routes et
lesquelles resteront vides. Le diagramme de Voronoï suffit bien à
représenter à la fois les éléments du viaire et du bâti et notre
dernière contrainte est comblée.

En réalité, dans ce modèle la ville est représentée par deux graphes
et le diagramme de Voronoï est uniquement employé en tant que point de
départ. Le premier, le graphe du bâti, a pour n\oe ud les centres des
parcelles alors que ses arêtes symbolisent les relations de
voisinage. Le second, le graphe viaire, a des arêtes représentant les
routes et des n\oe uds carrefour joignant plusieurs voies. Les
structures des graphe viaire et bâti sont donc entièrement fondées sur
le diagramme de Voronoï puisqu'il s'agit, respectivement, de
l'ensemble des arêtes et sommets de Voronoï et de sa triangulation de
Delaunay. REFORMULER

Il est essentiel de dissocier le polygone convexe qu'est la cellule de
Voronoï associée à une parcelle et la véritable empreinte cadastrale
de cette dernière. Une cellule représente l'influence d'une parcelle
dans l'espace urbain et possède comme seul point commun avec
l'empreinte son centre puisqu'il s'agit du générateur de la
cellule. Similairement, une arête peut indiquer qu'une voie passe
entre deux parcelles sans pour autant fournir ses coordonnées ou sa
courbure. Si l'on souhaite, dans un but infographique, générer une
image de notre ville à partir de ce modèle, un travail
d'interprétation est nécessaire et n'a pas été traîté à l'occasion de
ce projet. Un exemple est visible sur la figure \ref{fig:interp}.

\begin{figure}[!ht]

  \centering
  \subcaptionbox{}[0.9\linewidth][c]{
    \includegraphics[width=.3\linewidth]{images/voronoi-interp0.png}
  }

  \subcaptionbox{}[.3\linewidth][c]{
    \includegraphics[width=.3\linewidth]{images/voronoi-interp1.png}
  }
  \subcaptionbox{}[.3\linewidth][c]{
    \includegraphics[width=.3\linewidth]{images/voronoi-interp2.png}
  }
  \subcaptionbox{}[.3\linewidth][c]{
    \includegraphics[width=.3\linewidth]{images/voronoi-interp3.png}
  }

  \caption{Un diagramme de Voronoï trivial et trois interprétations possibles.}
  \label{fig:interp}
\end{figure}

\subsection{Potentialité}

Via le terme \textit{potentialité}, on souhaite exprimer l'opposition
entre deux types d'éléments : les \textit{potentiels} et les
\textit{construits}.

Un élément \textit{construit} est une parcelle ou une voie dont
l'existence physique est avérée. Il existe \textit{en dur} et affecte
ses alentours. L'ensemble des éléments construit forme la ville (voir
figure \ref{fig:construit}).

\begin{figure}
  \centering
  IMAGE
  \caption{Les éléments construits forment la ville.}
  \label{fig:construit}
\end{figure}

Un élément \textit{potentiel} peut être assimilé à une idée germant
dans l'esprit de l'urbaniste ; à une possibilité envisagée et
représentée de manière intangible. Un élément potentiel est par la
suite soit construit, soit ignoré et oublié. Il sert de prévision à
court-terme quant à l'avenir de la ville et guide sa morphogénèse. La
figure \ref{fig:potentiel} reprend la micro-ville de la figure
\ref{fig:construit} et laisse apparaître voies et parcelles
potentielles.

\begin{figure}
  \centering
  IMAGE A FAIRE
  \caption{Les éléments potentiels guident la croissance
    de la ville.}
  \label{fig:potentiel}
\end{figure}

Un élément potentiel, n'étant pas actif au sein de la ville et
appartenant uniquement au domaine du prévisionnel, n'a pas d'influence
sur les éléments construits. Par contre, la construction de nouveaux
éléments peut en dépendre \textit{e.g.} une route peut être construite
en conséquence à cette prévision, comme attirée par cette portentielle
future installation. Cette dualité dont les relations d'influence sont
clairement unidirectionnelles est inspirée du cycle réel
d'urbanisation que l'on pourrait grossièrement décomposer en ces
quelques étapes :

\begin{enumerate}
\item{Un urbaniste prévoit une nouvelle installation en bordure de
  ville;}
\item{Cette prévision attire la route;}
\item{La nouvelle route et l'installation potentielle attirent
  d'autres installations potentielles.}
\end{enumerate}
REFORMULER

L'essence du modèle est de placer des éléments potentiels en fonctions
de qualités internes au système puis de choisir lesquels véritablement
construire. Ce travail a été décomposé en trois mécanismes
distincts. ETOFFER

\subsection{Mécanismes}

\subsubsection{Automate cellulaire graphe}

La dynamique de croissance urbaine est décomposable sur deux axes. La
croissance horizontale décrit l'expansion spatiale de la ville dont
l'enveloppe grandit pour occuper plus de territoire tandis que la
croissance verticale correspond à l'augmentation des densités au sein
de la ville, souvent à partir d'un ou de plusieurs centres. Le
mécanisme cellulaire présenté ci-après émule la croissance verticale
et les variations de densité internes au système.

La densité de population est la quantité principale guidant
l'évolution de ce modèle. Même si le mécanisme proposé reste
trivialement simple, il guide les autres mécanismes du modèle : le
placement de nouvelles parcelles et le choix des routes à
construire. La ville évolue, de nouveaux bâtiments apparaissent,
d'autres sont rasés, les quartiers changent et le modèle doit être
capable de simuler ces changements. C'est bien sûr avec le principe
des automates cellulaires en tête que nous allons gérer cette
dynamique.

On discrétise la densité sur trois paliers : \textit{faible} ($f$),
\textit{moyenne} ($m$) et \textit{élevée} ($e$). L'état d'une parcelle
dépend des états de ses voisines. La matrice $A$ décrit des
coefficients d'affinité mettant en relation les différentes
densités.

\begin{equation}
A =
\bordermatrix{
    & f & m & e \cr
  f & 1 & 0.01 & 0 \cr
  m & 0.001 & 1.5 & 0.01 \cr
  e & 0 & 0.01 & 1.6
}
\end{equation}

Une valeur haute en $A_{ee}$ signifie par exemple que si une cellule a
de nombreux voisins de densité \textit{élevée} alors elle a une grande
probabilité de devenir elle-même \textit{élevée}. L'équation
\ref{eq:transition} permet de formaliser ce principe et fournit un
score $Ti(C)$ quantifiant l'éventualité pour une cellule $C$ de passer
à l'état $i$. $V_k(C)$ correspond au nombre de voisins de $C$ ayant
l'état $k$.

\begin{equation}
T_i(C) = \sum_{k \in \{f,m,e\}} V_k(C) E_{ik}
\label{eq:transition}
\end{equation}

Pour obtenir la probabilité de passage d'un état à un autre, on
normalise chaque score de transition et une roue de la fortune biaisée
se charge du choix.

\begin{equation}
P_i(C) = \frac{T_i(C)}{\sum_{k \in \{f,m,e\}} T_k(C)}
\label{eq:normalisation}
\end{equation}

Prenons un exemple trivial pour illustrer ce processus. On souhaite
évaluer l'état que pourrait prendre la cellule $C$, au centre du
quadrillage de la figure \ref{fig:ca-example}. Chacune des cellules en
bordure de cette figure sont voisines de $C$ selon le voisinage de
Moore. On commence par calculer les scores de transition.

\begin{figure}[!ht]
  \centering
  \begin{tikzpicture}

  \fill[densitylow] (0,0) rectangle (1,1);
  \fill[densitymedium] (1,0) rectangle (2,1);
  \fill[densityhigh] (2,0) rectangle (3,1);
  \fill[densitymedium] (0,1) rectangle (1,2);
  \fill[densitymedium] (1,1) rectangle (2,2);
  \fill[densityhigh] (2,1) rectangle (3,2);
  \fill[densitymedium] (0,2) rectangle (1,3);
  \fill[densityhigh] (1,2) rectangle (2,3);
  \fill[densityhigh] (2,2) rectangle (3,3);

  \draw[step=1,black] (0,0) grid (3,3);

  \draw[black,line width=2] (1,1) rectangle (2,2);
\end{tikzpicture}

  \caption{On cherche à calculer les probabilités transitionnelles
    pour la cellule centrale, $C$.}
  \label{fig:ca-example}
\end{figure}

\begin{align*}
T_f(C) &= V_f(C) A_{ff} + V_m(C) A_{fm} + V_e(C) A_{fe} \\
       &= 1 \times 1 + 3 \times 0.01 + 4 \times 0 \\
       &= 1.03
\end{align*}

\begin{align*}
T_m(C) &= V_f(C) A_{mf} + V_m(C) A_{mm} + V_e(C) A_{me} \\
       &= 1 \times 0.001 + 3 \times 1.5 + 4 \times 0.01 \\
       &= 4.541
\end{align*}

\begin{align*}
T_e(C) &= V_f(C) A_{ef} + V_m(C) A_{em} + V_e(C) A_{ee} \\
       &= 1 \times 0 + 3 \times 0.01 + 4 \times 1.6 \\
       &= 6.7
\end{align*}

On normalise ensuite les scores afin de sélectionner aléatoirement --
mais de façon biaisée -- le prochain état de $C$. Ici, on observe que
la cellule a de fortes chances de passer à la densité \textit{élevée}.

\begin{equation*}
P_f(C) = \frac{T_f(C)}{\sum_{k \in \{f,m,e\}} T_k(C)} = \frac{1.03}{1.03 + 4.541 + 6.7} = 0.084
\end{equation*}

\begin{equation*}
P_f(C) = \frac{T_f(C)}{\sum_{k \in \{f,m,e\}} T_k(C)} = \frac{1.03}{1.03 + 4.541 + 6.7} = 0.37
\end{equation*}

\begin{equation*}
P_f(C) = \frac{T_f(C)}{\sum_{k \in \{f,m,e\}} T_k(C)} = \frac{1.03}{1.03 + 4.541 + 6.7} = 0.546
\end{equation*}

Ce processus basé sur les affinités entre différentes classes évoque
le modèle de ségrégation de Schelling à la différence qu'ici, trois
\textit{populations} interagissent et qu'il n'y a pas de contrainte de
déménagement (dans son modèle, si une cellule passe de $A$ à $B$ alors
une autre doit passer de $B$ à $A$ ailleurs afin de conserver les
mêmes quantités de chaque type). Ainsi un dégradé discret de densité
apparaît comme dans une ville réelle.

Si l'on applique cette règle à un automate cellulaire classique, on
obtient les configurations visibles sur la figure \ref{fig:ac}. On
remarque deux incohérences séparant le système urbain simulé et une
ville. Premièrement, l'état de l'automate change drastiquement en
juste quelques itérations : au temps 25, la disposition de départ
n'est déjà plus discernable. Hors, la granularité temporelle d'une
telle simulation doit être fine afin de pouvoir prendre en compte
chaque modification locale du système pour qu'elle puisse se
répercuter sur le reste de l'automate. Deuxièmement, on observe
d'itération en itération que chaque cellule voit son état changer en
permanence -- ce qui est normal pour un automate cellulaire auquel on
n'a pas adjoint de règle supplémentaire. Il est donc important
d'associer à chaque cellule un élan favorisant la persistance de son
état selon son âge afin de ralentir la simulation et surtout d'éviter
les transitions constantes qui ne sont absolument pas fidèles à la
stabilité d'une ville réelle. Les fonctions sigmoïdes, fréquemment
employées en modélisation de systèmes complexes, sont idéales pour
exprimer en fonction du temps le pasage d'un seuil à un autre. La
sigmoïde classique (figure \ref{fig:sigmoide1}) varie de 0 à 1 par une
courbe caractéristique. On l'altère comme il est visible sur la figure
\ref{fig:sigmoide2} pour obtenir une fonction associant une
probabilité de changement d'état à l'âge de la cellule considérée. Le
facteur 0.02 permet d'adoucir la pente de la sigmoïde tandis que 350
sert à la décaler $f(x)$ de façon à pouvoir l'utiliser dans le domaine
positif.

\begin{figure}[H]
  \centering
  \subcaptionbox{$t = 0$}[.3\linewidth][c]{
    \includegraphics[width=.3\linewidth]{images/ca_0.png}
  }
  \subcaptionbox{$t = 25$}[.3\linewidth][c]{
    \includegraphics[width=.3\linewidth]{images/ca_25.png}
  }

  \subcaptionbox{$t = 50$}[.3\linewidth][c]{
    \includegraphics[width=.3\linewidth]{images/ca_50.png}
  }
  \subcaptionbox{$t = 100$}[.3\linewidth][c]{
    \includegraphics[width=.3\linewidth]{images/ca_100.png}
  }
  \caption{Quatre configurations de l'automate cellulaire. On y
    retrouve peu de similarités et les changements sont trop rapides.}
  \label{fig:ac}
\end{figure}

\begin{figure}[H]
  \centering
  \subcaptionbox{}[.3\linewidth][c]{
    \raisebox{3cm}{$f(x) = \frac{1}{ 1 + e^{-x}}$}
  }
  \subcaptionbox{}[.6\linewidth][c]{
    \includegraphics[width=.6\linewidth]{images/sigmoid.png}
  }
  \caption{Sigmoïde classique.}
  \label{fig:sigmoide1}
\end{figure}

\begin{figure}[H]
  \centering
  \subcaptionbox{}[.3\linewidth][c]{
    \raisebox{3cm}{$f(t) = \frac{1}{1 + e^{-0.02(t-350)}}$}
  }
  \subcaptionbox{}[.6\linewidth][c]{
    \includegraphics[width=.6\linewidth]{images/sigmoid-age.png}
  }
  \caption{Probabilité de changement d'état en fonction du temps.}
  \label{fig:sigmoide2}
\end{figure}

La figure \ref{fig:ac-stable} montre un automate cellulaire doté des
mêmes règles de transition et de la même configuration de départ mais
pour lequel l'âge des cellules est pris en compte lors de l'évaluation
de l'état suivant.

\begin{figure}[H]
  \centering
  \subcaptionbox{$t = 0$}[.3\linewidth][c]{
    \includegraphics[width=.3\linewidth]{images/sca_0.png}
  }
  \subcaptionbox{$t = 500$}[.3\linewidth][c]{
    \includegraphics[width=.3\linewidth]{images/sca_500.png}
  }
  \subcaptionbox{$t = 1000$}[.3\linewidth][c]{
    \includegraphics[width=.3\linewidth]{images/sca_1000.png}
  }

  \subcaptionbox{$t = 2000$}[.3\linewidth][c]{
    \includegraphics[width=.3\linewidth]{images/sca_2000.png}
  }
  \subcaptionbox{$t = 4000$}[.3\linewidth][c]{
    \includegraphics[width=.3\linewidth]{images/sca_4000.png}
  }
  \subcaptionbox{$t = 6000$}[.3\linewidth][c]{
    \includegraphics[width=.3\linewidth]{images/sca_6000.png}
  }
  \caption{Six configurations de l'automate cellulaire stabilisé.}
  \label{fig:ac-stable}
\end{figure}

Les exemples précédents permettent d'illustrer les règles de
transition et mettent en évidence un problème de stabilité et de
rythme à prendre en compte mais le principe même de cet exposé est de
se détacher de la régularité spatiale contraignante des automates
cellulaires et c'est à cet effet que l'on a présenté le diagramme de
Voronoï plus tôt. Dans notre diagramme de Voronoï, à la manière des
automates cellulaires graphes de O'Sullivan \cite{O'Sullivan2000},
chaque cellule voit son état varier en fonction de son voisinage ;
voisinage établit à partir de la topologie du diagramme, lui-même issu
des positions des centres des parcelles. Il est à noter qu'à la
différence de O'Sullivan, la couverture de l'espace est totale puisque
l'on ne représente pas les parcelles exactes mais leur cellule de
Voronoï et il est donc impossible que des espaces vides apparaissent
entre les cellules.

Le \textit{graphe du bâti} prend ici la forme duale du diagramme de
Voronoï, la triangulation de Delaunay, et décrit les relations de
voisinage. Un n\oe ud correspond au centre d'une parcelle et une arête
lie deux parcelles si elles sont voisines et partagent une arête. On
voit sur la figure \ref{fig:delaunay} que les parcelles en contact
sont liées. La configuration initiale contient autant de n\oe uds que
l'automate cellulaire classique mais leur positionnement est aléatoire
(bien que que homogène). Les quelques itérations du mécanisme
cellulaire appliqué au diagramme sont visibles sur la figure
\ref{fig:ac-voronoi} et dévoilent des résultats très similaires à la
version cellulaire classique.

\begin{figure}[H]
  \centering
  \includegraphics[width=.7\linewidth]{images/delaunay.png}
  \caption{}
  \label{fig:delaunay}
\end{figure}

\begin{figure}[H]
  \centering
  \subcaptionbox{$t = 0$}[.3\linewidth][c]{
    \includegraphics[width=.3\linewidth]{images/vca_0.png}
  }
  \subcaptionbox{$t = 500$}[.3\linewidth][c]{
    \includegraphics[width=.3\linewidth]{images/vca_500.png}
  }
  \subcaptionbox{$t = 1000$}[.3\linewidth][c]{
    \includegraphics[width=.3\linewidth]{images/vca_1000.png}
  }

  \subcaptionbox{$t = 2000$}[.3\linewidth][c]{
    \includegraphics[width=.3\linewidth]{images/vca_2000.png}
  }
  \subcaptionbox{$t = 4000$}[.3\linewidth][c]{
    \includegraphics[width=.3\linewidth]{images/vca_4000.png}
  }
  \subcaptionbox{$t = 6000$}[.3\linewidth][c]{
    \includegraphics[width=.3\linewidth]{images/vca_6000.png}
  }
  \caption{Six configurations du diagramme de Voronoï cellulaire.}
  \label{fig:ac-voronoi}
\end{figure}

Plus la simulation avance et plus le système est chargé de parcelles à
haute densité. Un lecteur averti pourrait argumenter que ce
comportement est bien différent d'une situation réelle : quel que soit
le taux de croissance d'une ville, elle ne finit jamais entièrement
remplie de grands immeubles et de centres commerciaux. Mais il faut
garder à l'esprit que le mécanisme présenté dans cette section ne gère
que la croissance verticale. Par la suite, la croissance horizontale
de la ville modifiera et étendra sa structure de façon à ce que les
nouvelles parcelles en bordures soient moins denses. C'est aussi l'un
des atouts du diagramme de Voronoï qui est une structure hautement
dynamique en terme de topologie contrairement à un automate
cellulaire.

La problématique étant d'étudier la coévolution de deux aspects
urbains, le viaire et le bâti, et non seulement l'évolution des
densités (qui ne sert que de support à l'essort de la ville), on a
préféré choisir une règle basique. Il est néanmoins tout à fait
possible d'utiliser par la suite des règles cellulaires plus
sophistiquées pour améliorer le réalisme de la simulation. On
pourrait, par exemple, prendre en compte les valeurs financières des
parcelles ou les classes sociales des habitants.

\subsubsection{Placement des éléments potentiels}

Le second mécanisme place de nouveaux éléments urbains en bordure de
la ville et est ainsi responsable de sa croissance horizontale. Ces
éléments sont les parcelles et les routes mais on se concentre
uniquement sur le placement des nouvelles parcelles puisque les routes
sont intégrées à leur structure.

Les parcelles et routes nouvellement placées bénéficient néanmoins
d'un statut spécial car elles sont considérées comme
\textit{potentielles} et se démarquent des éléments
\textit{construits} par leur impact sur l'évolution du système :

\begin{itemize}
\item{L'état d'une parcelle potentielle dépend de toutes ses voisines;}
\item{Une parcelle potentielle n'influence pas l'état des parcelles
  voisines construites;}
\item{Une parcelle potentielle peut accueillir des routes construites
  sur ses arêtes.}
\end{itemize}

Cette influence unidirectionnelle dans le mécanisme cellulaire permet
à la parcelle potentielle d'être prête et accordée à son environnement
proche si elle est construite par la suite sans pour autant que les
parcelles déjà construites ne soient influencées par une parcelle
intangible n'existant pas encore.

En quelques mots, le placement se déroule comme suit :

\begin{enumerate}
\item{On détermine les centres de la ville en fonction de la densité}
\item{On dépose sur un des centres une \textit{graine} mobile qui
  servira de centre à la nouvelle parcelle;}
\item{La graine se déplace sous l'influence des variables inhérentes à
  la ville;}
\item{Quand la graine stoppe son mouvement, on y crée la parcelle.}
\end{enumerate}

CENTRES?

Le déplacement de la graine est un processus pouvant potentiellement
prendre en compte de nombreuses variables. Dans la simulation
d'exemple que l'on décrit, seules la densité et le placement des
routes peuvent guider la graine, car ce sont les seules données
considérées. Cependant, on souhaite que le modèle soit extensible et
qu'il soit capable de supporter d'autres variables et contraintes : la
valeur des sols par exemple, ou bien la pente des zones envisagées ou
l'impossibilité de s'installer sur certains types de terrain (forêts,
plans d'eau). De cette idée de graine se déplaçant en fonction
d'influences diverses transpire un véritable aspect physique.

Pour rester en accord avec cet aspect physique, on emploie un champ de
vecteurs généré à partir de l'état du sytème urbain pour guider la
graine vers sa destination. Puisque de nombreux paramètres sont à
prendre en compte, on utilise un champ de vecteurs par paramètre que
l'on souhaite exprimer puis on les combine, ce qui permet de pondérer
l'impact de chaque donnée.

On veut évidemment que la densité soit prise en compte dans ce
placement. Dans une ville, les nouvelles installations ont
principalement tendance à se répartir sur les frontières de
l'enveloppe urbaine et à s'éloigner des centres, non pas par animosité
envers l'activité du centre-ville mais simplement par manque
d'espace. Pour traduire ce phénomène dans le modèle, un premier champ
fait donc pointer chacun de ses vecteurs vers la parcelle disposant de
la plus faible densité parmis les voisines de la parcelle sur laquelle
il est posé.

Il est naturel que les parcelles potentielles soient placées près des
installations routières existantes de façon à pouvoir communiquer avec
les grands axes routiers. À cet effet, on ajoute un champ pour lequel
chaque vecteur pointe vers la voie construite la plus proche.

Les nouvelles parcelles ne sont pas uniquement disposés en fonction de
variables internes à la ville car, souvent, l'environnement impose des
restrictions quant à la direction que l'expansion d'une ville va
prendre. Parmi ces contraintes, on pense aux zones non constructibles
comme les plans d'eau, les fôrets mais aussi aux fortes pentes. Un
champ de vecteurs est dédié à l'évitement de telles zones et repousse la
graine de ces obstacles. Hors de leur aire, l'influence est nulle.

EQUATION

\begin{equation*}
  I(x,y) = \alpha I_d(x,y) + \beta I_r(x,y) + \gamma I_o(x,y) + \dots
\end{equation*}

\begin{figure}[!ht]
  \centering
  \subcaptionbox{La ville.}[.4\linewidth][c]{
    \includegraphics[width=.4\linewidth]{images/vf-base.png}
  }
  \subcaptionbox{Le champ d'évasion de la densité.}[.4\linewidth][c]{
    \includegraphics[width=.4\linewidth]{images/vf-density.png}
  }

  \subcaptionbox{Le champ d'attraction des routes}[.4\linewidth][c]{
    \includegraphics[width=.4\linewidth]{images/vf-road.png}
  }
  \subcaptionbox{Le champ de répulsion des obstacles}[.4\linewidth][c]{
    \includegraphics[width=.4\linewidth]{images/vf-obstacle.png}
  }
  \caption{}
  \label{fig:fields}
\end{figure}

\begin{figure}[!ht]
  \centering
  \includegraphics[width=.8\linewidth]{images/vf-sum.png}
  \caption{}
  \label{fig:field-sum}
\end{figure}

DEPLACEMENT GRAINE. VITESSE. ARRET.

Les champs de vecteurs décrit ci-dessus servent uniquement
d'illustrations et la généralité du concept de ce mécanisme permet à
l'utilisateur du modèle d'employer les données qu'il juge nécessaires
afin de mettre en avant certaines caractéristiques d'un système
urbain.

\subsubsection{Construction des éléments potentiels}

ROUTES

PARCELLES

\section{Construction}

Notre système urbain est décrit sur deux niveaux; plus précisément,
par deux graphes. On distingue le graphe parcellaire du graphe viaire
car, bien qu'ils soient étroitement liés, ils représentent des couches
différentes du réseau urbain. Leur topologie est directement issue du
diagramme de Voronoï et l'on décrit dans cette section les
problématiques pratiques rencontrées lors de la transition de cette
structure spatiale de base aux structures relationnelles que sont les
graphes.

\section{Mesures}

HEUUUUUUUU...

DEGRE DES CARREFOURS

ELOIGNEMENT DE LA DENSITE PAR RAPPORT AU CENTRE GEOMETRIQUE

ELOIGNEMENT DE LA DENSITE PAR RAPPORT AUX CENTRES DENSITAIRES

TAILLE DES PARCELLES ? (BIAIS AU BORD)

ACCESSIBILITE ?

DIAMETRE ?

CENTRALITE ?

\section{Conclusion}

RESUME

CONSTAT

PERSPECTIVES (DYNAMIQUES INTERNES, REALISME, INTERPRETATION, VRAIE VILLE)

\printbibliography

\end{document}
