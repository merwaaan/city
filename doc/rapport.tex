\documentclass[10pt]{article}

\usepackage[utf8]{inputenc}
\usepackage[francais]{babel}
\usepackage{multicol}
\usepackage{subfig}

\usepackage{tikz}
\usetikzlibrary{calc}

\usepackage[margin=1.2in]{geometry}
\setlength{\columnsep}{2pc}

\usepackage[natbib=true,sorting=none,style=numeric,url=false,doi=false,isbn=false]{biblatex}
%\addbibresource{references}
\bibliography{references}

\title{TITRE}
\author{Merwan Achibet}
\date{}

\begin{document}

\maketitle

\begin{abstract}
abstract abstract abstract abstract abstract abstract abstract
abstract abstract abstract abstract abstract abstract abstract
abstract abstract abstract abstract abstract abstract abstract
abstract abstract abstract abstract abstract abstract abstract
abstract abstract abstract abstract abstract abstract abstract
abstract abstract abstract abstract abstract abstract abstract
abstract abstract abstract abstract abstract abstract abstract
abstract abstract abstract abstract abstract abstract abstract
abstract abstract abstract abstract abstract abstract abstract
\end{abstract}

\begin{multicols}{2}

\section{Introduction}

\section{\'Etat de l'art}

La modélisation de systèmes complexes est longtemps uniquement passée
par l'usage de méthodes mathématiques; typiquement, des systèmes
d'équations différentielles. Ces techniques permettent de décrire des
lois d'évolution et d'observer, ainsi que de prédire par
extrapolation, le comportement de phénomènes du réel. Dans le cas de
systèmes prenant en compte un vaste jeu de paramètres, cette approche
peut néanmoins se révéler délicate à employer. Plus intrinsèquement,
même si une telle modélisation est basée sur des observations ancrées
dans la réalité, il s'agit d'une représentation conceptuelle d'un
problème et aucune mimique des mécaniques sous-jacentes ne s'opère.

D'un point de vue historique, les prémices de l'informatique moderne
et d'un tout autre paradigme de modélisation sont à attribuer aux
esprits du milieu du vingtième siècle. Alan Turing introduit en 1936
la machine éponyme qui, bien que purement théorique, possède un module
de contrôle ainsi qu'une mémoire et peut donc exécuter toutes sortes
de procédures. Cette démarche se démarque de l'approche mathématique
et semble plus humaine; on ne résout pas un problème en utilisant des
fonctions associant une quantité à un résultat mais on agit
véritablement sur ses données. L'idée de base de Turing était
d'ailleurs d'assimiler le fonctionnement de sa machine au travail
d'une personne remplissant les cases d'un tableau infini CITATION?.

Entraîné par cette mouvance procédurale et en réaction aux réseaux de
neurones de McCulloch et Pitts, John von Neumann introduit le terme
d'automate en 1946. Simplement, un automate est une machine qui opère
sur des données fournies en entrée en fonction de règles internes
prédéfinies. On choisit de se concentrer sur une sous-catégorie
d'automates, les automates finis à états, changeant leur
représentation interne selon des règles de transition. John von
Neumann et Stanislaw Ulam joignent leurs travaux pour concevoir
l'automate cellulaire : un système comprenant un ensemble d'automates
à états spatialement localisés (typiquement sous forme de grille) et
interconnectés en fonction de leur proximité. Les entrées de chaque
automate correspondent alors aux états des automates voisins et de
cette organisation se dégagent de fortes relations
d'interdépendance. Le jeu de la vie de Conway en est un exemple
classique. La simplicité de ses règles, mise en contraste avec la
variété des configurations engendrées, témoigne de la richesse des
automates cellulaires CITATION?.

Les automates cellulaires ont depuis été extensivement étudiés et sont
appliqués à de nombreux phénomènes biologiques, physiques et sociaux
\cite{Ganguly}. La motivation d'Ulam lors de leur conception était
d'ailleurs de modéliser la croissance de cristaux. On peut aussi citer
en exemple la simulation de la dynamique de fluides \cite{Frisch1986},
de la croissance de tumeurs \cite{Kansal2000}, de l'évolution
d'épidémies \cite{Fu2003}, de la ségrégation de population
CITATION-SCHELLING. Leur caractère spatial laisse supposer qu'ils sont
particulièrement adaptés aux applications géographiques, et plus
précisément, urbaines. Ils ne fûrent paradoxalement pas immédiatement
exploités à cet effet et c'est seulement suite à un article de Waldo
Tobler, en 1975, que le rapprochement entre les automates cellulaires
et le domaine de la géographie apparaît clairement \cite{Tobler1975}.

Une idée très exploitée dans ce domaine est d'associer un potentiel de
transition à chaque cellule vers tous les états qu'elles peuvent
prendre. Dans les modèles déterministes, la transition vers l'état à
plus haut potentiel est appliquée tandis que dans les modèles
stochastiques, un tirage aléatoire biaisée a lieu. Le potentiel d'une
cellule à passer à un nouvel état est déterminé en fonction de
paramètres propres à la simulation. Peuvent être pris en compte
l'élévation du terrain, la densité de population, la proximité d'axes
routiers, la proximité de centres urbains, l'âge des parcelle, leur
valeur; en fait, toute combinaison d'attributs relatifs à un réseau
urbain. Par exemple, dans une simulation représentant les différents
types d'usage, le passage d'une cellule à l'état \textit{résidentiel}
pourrait dépendre de la proximité des commerces et des routes et de
l'éloignement des zones industrielles. Bien sûr, un nombre élevé de
paramètres à prendre en compte requiert un couplage fin et l'impact de
chaque variable peut être pondéré. Puisque les variations
individuelles de paramètres n'émergent pas de manière transparente à
la surface de la simulation, les modèles urbains basés sur des
automates cellulaires doivent être finement calibrés et leur réalisme
est un défi en soi. Pour contourner ce problème, Yeh et Li prône
l'usage d'un réseau de neurones pour pondérer chaque paramètre à
partir de l'analyse de données cartographiques historiques
\cite{Yeh2002}.

Il est important de noter que la simplicité des règles régissant le
fonctionnement d'un automate cellulaire strict s'oppose à la qualité
de la simulation, notamment dans le cadre de modèles spécifiques
\cite{Torrens2001}. Dans ce cas, une prise de liberté quant aux
formalisme originel est autorisée, voire nécessaire, pour obtenir des
résultats satisfaisants \cite{White1998}.

La première limite que le formalisme cellulaire de base impose est la
discrétisation des états que chaque cellule peut adopter. Même si
cette caractéristique fait partie intégrante des particularités qui
confèrent aux automates cellulaires leur simplicité, la description de
quantités pouvant arborer un large éventail de valeurs est alors
impossible. Plus concrètement, il est aisé de catégoriser les cellules
d'un espace selon le fait, par exemple, qu'elles contiennent des
installations humaines ou non (état booléen)
\cite{Benguigui2004,Cornu} ou de façon plus sophistiquée, en fonction
de leur type d'usage (\textit{résidentiel}, \textit{commercial} et
\textit{industriel} \cite{Lechner} et plus
\cite{Dubos-Paillard203}). Représenter des quantités réelles est plus
délicat. Pour symboliser la densité d'un ensemble urbain, Semboloni
utilise par exemple un automate cellulaire de dimension trois dans
lequel plus une pile de cellules occupées est haute et plus la zone
représentée est peuplée \cite{Semboloni2000}. Plus généralement, on
peut s'autoriser à représenter l'état d'une cellule par un vecteur
contenant des valeurs réelles. Des règles de transitions adaptées et
mesurées sont alors à mettre en place.

L'homogénéité d'un automate cellulaire fait partie intégrante de sa
définition originelle : en mettant de côté l'état qu'elles adoptent,
toutes les cellules sont identiques en forme et en structure de
voisinage. Dans le cadre de notre problématique, cette approche est
limitante car, dans une ville, les parcelles ne sont que rarement
identiques et alignées. Similairement, la notion de voisinage est
clairement à redéfinir. Pour des problèmes classiques, les voisinages
de von Neumann et de Moore sont régulièrement utilisés mais la
relation par contiguité qu'ils décrivent ne convient pas à la
représentation des liens de dépendance à plus grande échelle se
développant dans un système urbain. Le positionnement d'un bâtiment
résidentiel dans une ville se base évidemment sur le voisinage direct
des zones envisagées (on veut ajouter une maison dans un quartier
résidentiel) mais il faut aussi prendre en compte les alentours plus
distants (la centrale thermique se trouvant à 500 mètres du site peut
poser problème). Une solution possible est d'étendre les voisinages de
von Neumann et de Moore tout en conservant leur forme
caractéristique. + AUTOMATE CELLULAIRE GRAPHE (THESE O'SULLIVAN)
\cite{O'Sullivan2000,0'Sullivan2001}.

Une prise de liberté quant à l'aspect temporel est aussi
envisageable. Un automate cellulaire strict est synchrone, i.e. les
changements d'état de toutes les cellules s'effectuent
simultanément. Si le choix était fait de mettre à jour chaque état de
façon asynchrone, le comportement de l'automate en serait lourdement
modifié. Par exemple, les qualités auto-réplicatives de certaines
entités du jeu de la vie ne seraient pas garanties. Il est pourtant
légitime de se questionner sur la validité d'un tel choix dans une
simulation urbaine, premièrement parce qu'une ville est un système
complexe et désordonné, deuxièmement parce les processus qui s'y
déroule sont réglés sur différentes échelles temporelles.

Les automates cellulaires ne sont pas l'unique moyen de modéliser la
croissance urbaine. Plusieurs simulations existantes sont des systèmes
multi-agent \cite{Lechnera,Lechner2004}. Dans ces cas, un agent est
assimilé à un promoteur immobilier et peut acheter des terres, les
vendre, les développer ou changer leur type. Les actions qu'il
entreprent sont évaluées en fonction de l'impact sur la ville
(changement de la valeur immobilière, avis de la population) et des
réglementations locales, afin d'éviter toute configuration
illégale.

D'autres solutions s'éloignant des systèmes complexes et penchant du
côté de la génération procédurale de contenu existent. Souvent, le
domaine d'application de telles méthodes est l'infographie, le cinéma
et le jeu vidéo et l'objectif est de construire de manière automatique
une ville visuellement réaliste sans se soucier de son caractère
fonctionnel. Usuellement, la première étape est de générer un réseau
viaire complet puis de placer le bâti en subdivisant récursivement les
niches vides formées par les routes. Dans Citygen CITATION, un point
$p$ de l'espace est aléatoirement choisi puis on calcule un ensemble
de plusieurs routes raccordant $p$ au réseau routier existant en
faisant varier leur déviation angulaire et un paramètre de bruit; la
route finale est celle pour laquelle la variation d'altitude est la
plus faible. CityEngine CITATION utilise un L-System dont les règles
permettent de reproduire les différents motifs quadrillés, radiaux et
organiques que l'on trouve dans une ville. La nature récursive des
L-Systems permet à ces motifs de se combiner et les résultats sont
saisissants. Dans une autre simulation CITATION, le tracé des routes
suit les \textit{hyperstreamlines} formées par un champ de
vecteurs. Ce champ est calculé par combinaison de plusieurs autres
champs de vecteurs, chacun représentant des contraintes de direction
particulières telles que les zone interdites (eau, espaces verts),
l'altitude et la densité de population. Ces techniques sont
intrinsèquement géométriques, et comme précisé plus haut, le résultat
est purement visuel, mais elles représentent une source d'inspiration
à ne pas négliger.

Bien que les automates cellulaires soient couramment utilisés pour
simuler le traffic routier (dans leur version une dimension CITATION
ou deux dimensions CITATION), ils s'accordent peu avec la construction
même d'un réseau viaire. Dans les simulations cellulaires urbaines, le
positionnement des routes a souvent un impact sur le développement des
cellules mais le réseau est fourni dés le début et reste
fixe. DEUX COUCHES ?

Dans les modèles à base d'agents, une solution est de mettre en place,
en plus des agents promoteurs, des agents traceurs de route chargés de
connecter DETAILS

L'un des seuls modèles gérant à la fois l'évolution du réseau viaire
et du bâti est présenté par Weber \cite{Weber2009} et n'emploie pas
d'automate cellulaire. Le principe est le suivant : à chaque
agrandissement du réseau urbain, on crée plusieurs routes virtuelles
en suivant des règles géométriques précises (allongement des voies
existantes, limitation du degré des carrefours à 4, l'angle entre
chaque rue tend vers 90 degrès). Parmi les $n$ routes générées, une
seule sera construite. Pour la choisir, des agents virtuels arpentent
ces routes afin d'évaluer laquelle sera la plus bénéfique au réseau.

\end{multicols}

%% \begin{figure}[h]

%%   \centering

%%   \subfloat[Automate cellulaire simple]{

%%     \begin{tikzpicture}
%%       \draw[step=1,gray] (0,0) grid (3,3);

%%       \node at (0.5,0.5) {G};
%%       \node at (1.5,0.5) {H};
%%       \node at (2.5,0.5) {I};
%%       \node at (0.5,1.5) {D};
%%       \node at (1.5,1.5) {E};
%%       \node at (2.5,1.5) {F};
%%       \node at (0.5,2.5) {A};
%%       \node at (1.5,2.5) {B};
%%       \node at (2.5,2.5) {C};
%%     \end{tikzpicture}

%%   }

%%   \subfloat[Automate cellulaire graphe]{

%%     \begin{tikzpicture}
%%       \draw (0.5,0.5) node[circle,draw] (g) {G};
%%       \draw (1.5,0.5) node[circle,draw] (h) {H};
%%       \draw (2.5,0.5) node[circle,draw] (i) {I};
%%       \draw (0.5,1.5) node[circle,draw] (d) {D};
%%       \draw (1.5,1.5) node[circle,draw] (e) {E};
%%       \draw (2.5,1.5) node[circle,draw] (f) {F};
%%       \draw (0.5,2.5) node[circle,draw] (a) {A};
%%       \draw (1.5,2.5) node[circle,draw] (b) {B};
%%       \draw (2.5,2.5) node[circle,draw] (c) {C};

%%       \draw (a) -- (b);
%%       \draw (a) -- (e);
%%       \draw (a) -- (d);
%%       \draw (b) -- (c);
%%       \draw (b) -- (f);
%%       \draw (b) -- (e);
%%       \draw (b) -- (d);
%%       \draw (b) -- (f);
%%       \draw (c) -- (f);
%%       \draw (d) -- (e);
%%       \draw (d) -- (h);
%%       \draw (d) -- (g);
%%       \draw (e) -- (c);
%%       \draw (e) -- (f);
%%       \draw (e) -- (i);
%%       \draw (e) -- (h);
%%       \draw (e) -- (g);
%%       \draw (f) -- (i);
%%       \draw (f) -- (h);
%%       \draw (g) -- (h);
%%       \draw (h) -- (i);
%%     \end{tikzpicture}

%%   }

%%   \caption{}
%%   \label{}

%% \end{figure}

\printbibliography

\end{document}
