\documentclass[10pt]{article}

\usepackage[utf8]{inputenc}
\usepackage[francais]{babel}
\usepackage{multicol}

\usepackage[margin=1.2in]{geometry}
\setlength{\columnsep}{2pc}

\usepackage[natbib=true,style=alphabetic]{biblatex}
\addbibresource{references}

\title{TITRE}
\author{Merwan Achibet}
\date{}

\begin{document}

\maketitle

\begin{abstract}
abstract abstract abstract abstract abstract abstract abstract
abstract abstract abstract abstract abstract abstract abstract
abstract abstract abstract abstract abstract abstract abstract
abstract abstract abstract abstract abstract abstract abstract
abstract abstract abstract abstract abstract abstract abstract
abstract abstract abstract abstract abstract abstract abstract
abstract abstract abstract abstract abstract abstract abstract
abstract abstract abstract abstract abstract abstract abstract
abstract abstract abstract abstract abstract abstract abstract
abstract abstract abstract abstract abstract abstract abstract
\end{abstract}

\begin{multicols}{2}

\section{Introduction}

\section{\'Etat de l'art}

La modélisation de systèmes complexes est longtemps uniquement passée
par l'usage de méthodes mathématiques; typiquement, des systèmes
d'équations différentielles. Ces techniques permettent de décrire des
lois d'évolution et d'observer, ainsi que de prédire par
extrapolation, le comportement de phénomènes du réel. Dans le cas de
systèmes prenant en compte un vaste jeu de paramètres, cette approche
peut néanmoins se révéler délicate à employer. Plus intrinsèquement,
même si une telle modélisation est basée sur des observations ancrées
dans la réalité, il s'agit d'une représentation conceptuelle d'un
problème et aucune mimique des mécaniques sous-jacentes ne s'opère.

D'un point de vue historique, les prémices de l'informatique moderne
et d'un tout autre paradigme de modélisation sont à attribuer aux
esprits du milieu du vingtième siècle. Alan Turing introduit en 1936
la machine éponyme qui, bien que purement théorique, est capable
d'éxécuter des procédures. Cette démarche se démarque de l'approche
mathématique et semble plus humaine; on ne résout pas un problème en
utilisant des fonctions associant une quantité à un résultat, mais on
agit véritablement sur les données du problème. L'idée de base de
Turing était d'ailleurs d'assimiler le fonctionnement de sa machine au
travail d'une personne remplissant les cases d'un tableau
infini. Entraîné par cette mouvance procédurale et en réaction aux
réseaux de neurones de McCulloch et Pitts, John von Neumann introduit
le terme d'automate en 1946.

Simplement, un automate est une machine qui, à partir de données
fournies en entrées, produit des données en sortie et ce, en fonction
de règles internes prédéfinies. On choisit de se concentrer sur une
sous-catégorie d'automates, les automates finis à états, changeant
leur représentation interne selon des règles de transition. John von
Neumann et Stanislaw Ulam joignent leurs travaux pour concevoir le
premier automate cellulaire. Un automate cellulaire est un système
comprenant un ensemble d'automates finis spatialement localisés,
typiquement sous forme de grille, et interconnectés en fonction de
leur proximité. Les entrées de chaque automate correspondent alors aux
états des automates voisins et de fortes relations d'interdépendance
se dégagent de cette organisation. Le jeu de la vie de Conway CITATION
en est un exemple classique. La simplicité de ses règles, mise en
contraste avec la variété des résultats produits témoigne de la
richesse des automates cellulaires.

Les automates cellulaires ont depuis été largement étudiés et servent
de support à de nombreuses modélisations de phénomènes biologiques et
physiques \cite{Ganguly}. La motivation d'Ulam lors de la conception
des automates cellulaires était d'ailleurs de modéliser la croissance
de cristaux. Ils sont aussi utilisés, entre autres, pour simuler la
dynamique de fluides \cite{Frisch1986}, la croissance de tumeurs
CITATION, les dynamiques proies-prédateurs CITATION.

Les automates cellulaires, notamment de par leur caractère spatial,
semblent particulièrement appropriés aux applications géographiques,
et plus précisément, urbaines. Paradoxalement, ils ne fûrent pas
immédiatement exploités à cet effet et c'est seulement suite à un
article de Waldo Tobler, en 1975, que le rapprochement entre les
automates cellulaires et le domaine de la géographie apparaît
clairement \cite{Tobler1975}.

Il est important de remarquer que la simplicité des règles régissant
le fonctionnement d'un automate cellulaire strict s'oppose à la
qualité de la modélisation, notamment lors de la conception de modèles
spécifiques \cite{Torrens2001}. Dans ce cas, une prise de liberté
quant aux formalisme originel est autorisée, voire nécessaire, pour
obtenir des résultats réalistes \cite{White1998}.

La première limite que le formalisme de base impose est la
discrétisation des états que chaque cellule peut adopter. Même si
cette caractéristique fait partie intégrante des particularités qui
confèrent aux automates cellulaires leur simplicité, la description de
quantités pouvant arborer un large éventail de valeurs est alors
impossible. Plus concrètement, il est aisé de catégoriser les cellules
d'un espace selon le fait qu'elles contiennent des installations
humaines ou non (état booléen) ou de façon plus sophistiquée en
fonction de leur usage terrestre (résidentiel, commercial ou
industriel CITATION et plus CITATION). Par contre, la représentation
de données ??? (densité, qualité des sols, valeur) BLABLA

L'homogénéité d'un automate cellulaire fait partie intégrante de sa
définition de base. En mettant de côté l'état qu'elles adoptent,
toutes les cellules sont identiques en forme et en structure de
voisinage. Dans le cadre de notre problématique, cette approche est
limitante car, dans une ville, les parcelles ne sont que rarement
identiques et alignées. TESSELATION VORONOI / TRIANGULATION DELAUNAY

De la même façon, la notion de voisinage est clairement à
redéfinir. Pour des problèmes classiques, les voisinages de von
Neumann et de Moore sont régulièrement utilisés mais la relation par
contiguité qu'ils décrivent ne convient pas à la représentation des
liens de dépendance à plus grande échelle se développant dans un
système urbain. Par exemple, l'attractivité d'une parcelle pour un
usage résidentiel peut paraître bonne si un voisinage de Moore indique
que les blocs avoisinants sont eux aussi résidentiels, mais ignore le
fait que le pôle industriel de la ville se situe à trois
blocs de là. Une solution possible est d'étendre les voisinages de von
Neumann et de Moore tout en conservant leur forme caractéristique
CITATION. Une autre idée est d'utiliser un automate cellulaire graphe
pour représenter un espace. Les graphes permettent de \cite{0'Sullivan2001}.

Dans les automates cellulaires classiques, la mise à jour de chaque
cellule en fonction de ses voisins est synchrone; elles sont toutes
mises à jour simultanément i.e. l'état suivant de chaque cellule est
déterminé avant le changement d'état ait lieu, afin d'éviter qu'elles
soient désynchronisées. Dans une ville par contre, où une cellule
représente une entité atomique, les changements d'états pouvant
s'opérer (quelle que soit la signification donné aux états) se font de
manière asynchrone. En effet, on ne peut imaginer que dans une ville
les modifications de l'espace urbains se fassent toutes au même
instant.

\end{multicols}

\printbibliography

\end{document}
