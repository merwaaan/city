\documentclass[10pt,twocolumn]{article}

\usepackage[utf8]{inputenc}
\usepackage[francais]{babel}

\usepackage{biblatex}
\addbibresource{references}

\title{Co-évolution du réseau viaire et du bâti d'une ville}
\author{Merwan Achibet}
\date{}

\begin{document}

\maketitle

\begin{abstract}

\end{abstract}

\section{Introduction}

\section{\'Etat de l'art}

La modélisation de systèmes complexes est longtemps uniquement passée
par l'usage de méthodes mathématiques, typiquement des systèmes
d'équations différentielles. Ces techniques permettent de décrire des
lois d'évolution et d'observer, ainsi que de prédire par
extrapolation, le comportement de phénomènes du réel. Cette approche
peut se révéler délicate dans le cas de systèmes prenant en compte un
vaste jeu de paramètres. Plus intrinsèquement, même si une telle
modélisation est basée sur des observations ancrées dans la réalité,
il s'agit d'une représentation conceptuelle d'un problème et aucune
mimique des mécaniques sous-jacentes ne s'opère.

D'un point de vue historique, les prémices de l'informatique moderne
et d'un tout autre paradigme de modélisation sont à attribuer à Alan
Turing qui, en 19?? introduit la machine éponyme. DETAILS

Par définition, un automate est une machine qui, à partir de données
fournies en entrées, produit des données en sortie et ce, en fonction
de règles internes prédéfinies. Un automate fini

Par la suite, John von Neumann et Stanislaw Ulam joignent leurs
travaux pour concevoir le premier automate cellulaire. Un automate
cellulaire est un système comprenant un ensemble d'automates finis
spatialement localisés, typiquement sous forme de grille, et
interconnectés en fonction de leur proximité. Les entrées de chaque
automate correspondent alors aux états des automates voisins et de
fortes relations d'interdépendance se dégagent de cette
organisation. Le jeu de la vie CITATION est un exemple bien connu
d'automate cellulaire. La simplicité de ses règles, mise en contraste
avec la variété des résultats produits témoigne de la richesse des
automates cellulaires.

Les automates cellulaires ont depuis été largement étudiés et servent
de support à de nombreuses modélisations de phénomènes du réel. Dans
un cadre physique, ils peuvent être utilisés pour simuler l'écoulement
de fluides ou bien la diffusion de gaz. Dans une optique plus humaine,
ils se révèlent aussi très adaptés à la modélisation de trafic routier
ou de la croissance urbaine -- notre sujet.

Paradoxalement, l'application d'automates cellulaires à des problèmes
géographiques ne fût pas immédiate. TOBLER

Bien que les automates cellulaires bénéficient de qualités évidentes,
leur simplicité peut se montrer contraignante lors de la conception de
modèles spécifiques. Dans ce cas, une prise de liberté quant aux
formalisme originel est autorisée, voire nécessaire, pour obtenir des
résultats réalistes.

La première limite que le formalisme de base impose est la
discrétisation des états que chaque cellule peut prendre. En effet,
même si cette caractéristique fait partie intégrante des
particularités qui confère aux automates cellulaires leur simplicité,
la description de quantités pouvant arborer un large éventail de
valeurs est fastidieuse.

Une autre contrainte, en liaison directe avec notre sujet, est
l'organisation en grille des cellules. Cette régularité qui, une fois
de plus, facilite l'emploi des AC, a un prix et limite clairement la
représentation du réel. Selon l'échelle d'une simulation urbaine, on
peut vouloir représenter au plus bas niveau un batîment, un quartier
ou bien même une zone de plusieurs kilomètres de côté, mais dans
chacun de ces cas les entités décrites ne sont que rarement carrées et
alignées dans une situation réelle. BLABLA

Dans les automates cellulaires classiques, la mise à jour de chaque
cellule en fonction de ses voisins est synchrone; elles sont toutes
mises à jour simultanément i.e. l'état suivant de chaque cellule est
déterminé avant le changement d'état ait lieu, afin d'éviter qu'elles
soient désynchronisées. Dans une ville par contre, où une cellule
représente une entité atomique, les changements d'états pouvant
s'opérer (quelle que soit la signification donné aux états) se font de
manière asynchrone. En effet, on ne peut imaginer que dans une ville
les modifications de l'espace urbains se fassent toutes au même
instant.

\printbibliography

\end{document}
